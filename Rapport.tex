
% Default to the notebook output style

    


% Inherit from the specified cell style.




    
\documentclass[11pt]{article}

    
    
    \usepackage[T1]{fontenc}
    % Nicer default font (+ math font) than Computer Modern for most use cases
    \usepackage{mathpazo}

    % Basic figure setup, for now with no caption control since it's done
    % automatically by Pandoc (which extracts ![](path) syntax from Markdown).
    \usepackage{graphicx}
    % We will generate all images so they have a width \maxwidth. This means
    % that they will get their normal width if they fit onto the page, but
    % are scaled down if they would overflow the margins.
    \makeatletter
    \def\maxwidth{\ifdim\Gin@nat@width>\linewidth\linewidth
    \else\Gin@nat@width\fi}
    \makeatother
    \let\Oldincludegraphics\includegraphics
    % Set max figure width to be 80% of text width, for now hardcoded.
    \renewcommand{\includegraphics}[1]{\Oldincludegraphics[width=.8\maxwidth]{#1}}
    % Ensure that by default, figures have no caption (until we provide a
    % proper Figure object with a Caption API and a way to capture that
    % in the conversion process - todo).
    \usepackage{caption}
    \DeclareCaptionLabelFormat{nolabel}{}
    \captionsetup{labelformat=nolabel}

    \usepackage{adjustbox} % Used to constrain images to a maximum size 
    \usepackage{xcolor} % Allow colors to be defined
    \usepackage{enumerate} % Needed for markdown enumerations to work
    \usepackage{geometry} % Used to adjust the document margins
    \usepackage{amsmath} % Equations
    \usepackage{amssymb} % Equations
    \usepackage{textcomp} % defines textquotesingle
    % Hack from http://tex.stackexchange.com/a/47451/13684:
    \AtBeginDocument{%
        \def\PYZsq{\textquotesingle}% Upright quotes in Pygmentized code
    }
    \usepackage{upquote} % Upright quotes for verbatim code
    \usepackage{eurosym} % defines \euro
    \usepackage[mathletters]{ucs} % Extended unicode (utf-8) support
    \usepackage[utf8x]{inputenc} % Allow utf-8 characters in the tex document
    \usepackage{fancyvrb} % verbatim replacement that allows latex
    \usepackage{grffile} % extends the file name processing of package graphics 
                         % to support a larger range 
    % The hyperref package gives us a pdf with properly built
    % internal navigation ('pdf bookmarks' for the table of contents,
    % internal cross-reference links, web links for URLs, etc.)
    \usepackage{hyperref}
    \usepackage{longtable} % longtable support required by pandoc >1.10
    \usepackage{booktabs}  % table support for pandoc > 1.12.2
    \usepackage[inline]{enumitem} % IRkernel/repr support (it uses the enumerate* environment)
    \usepackage[normalem]{ulem} % ulem is needed to support strikethroughs (\sout)
                                % normalem makes italics be italics, not underlines
    

    
    
    % Colors for the hyperref package
    \definecolor{urlcolor}{rgb}{0,.145,.698}
    \definecolor{linkcolor}{rgb}{.71,0.21,0.01}
    \definecolor{citecolor}{rgb}{.12,.54,.11}

    % ANSI colors
    \definecolor{ansi-black}{HTML}{3E424D}
    \definecolor{ansi-black-intense}{HTML}{282C36}
    \definecolor{ansi-red}{HTML}{E75C58}
    \definecolor{ansi-red-intense}{HTML}{B22B31}
    \definecolor{ansi-green}{HTML}{00A250}
    \definecolor{ansi-green-intense}{HTML}{007427}
    \definecolor{ansi-yellow}{HTML}{DDB62B}
    \definecolor{ansi-yellow-intense}{HTML}{B27D12}
    \definecolor{ansi-blue}{HTML}{208FFB}
    \definecolor{ansi-blue-intense}{HTML}{0065CA}
    \definecolor{ansi-magenta}{HTML}{D160C4}
    \definecolor{ansi-magenta-intense}{HTML}{A03196}
    \definecolor{ansi-cyan}{HTML}{60C6C8}
    \definecolor{ansi-cyan-intense}{HTML}{258F8F}
    \definecolor{ansi-white}{HTML}{C5C1B4}
    \definecolor{ansi-white-intense}{HTML}{A1A6B2}

    % commands and environments needed by pandoc snippets
    % extracted from the output of `pandoc -s`
    \providecommand{\tightlist}{%
      \setlength{\itemsep}{0pt}\setlength{\parskip}{0pt}}
    \DefineVerbatimEnvironment{Highlighting}{Verbatim}{commandchars=\\\{\}}
    % Add ',fontsize=\small' for more characters per line
    \newenvironment{Shaded}{}{}
    \newcommand{\KeywordTok}[1]{\textcolor[rgb]{0.00,0.44,0.13}{\textbf{{#1}}}}
    \newcommand{\DataTypeTok}[1]{\textcolor[rgb]{0.56,0.13,0.00}{{#1}}}
    \newcommand{\DecValTok}[1]{\textcolor[rgb]{0.25,0.63,0.44}{{#1}}}
    \newcommand{\BaseNTok}[1]{\textcolor[rgb]{0.25,0.63,0.44}{{#1}}}
    \newcommand{\FloatTok}[1]{\textcolor[rgb]{0.25,0.63,0.44}{{#1}}}
    \newcommand{\CharTok}[1]{\textcolor[rgb]{0.25,0.44,0.63}{{#1}}}
    \newcommand{\StringTok}[1]{\textcolor[rgb]{0.25,0.44,0.63}{{#1}}}
    \newcommand{\CommentTok}[1]{\textcolor[rgb]{0.38,0.63,0.69}{\textit{{#1}}}}
    \newcommand{\OtherTok}[1]{\textcolor[rgb]{0.00,0.44,0.13}{{#1}}}
    \newcommand{\AlertTok}[1]{\textcolor[rgb]{1.00,0.00,0.00}{\textbf{{#1}}}}
    \newcommand{\FunctionTok}[1]{\textcolor[rgb]{0.02,0.16,0.49}{{#1}}}
    \newcommand{\RegionMarkerTok}[1]{{#1}}
    \newcommand{\ErrorTok}[1]{\textcolor[rgb]{1.00,0.00,0.00}{\textbf{{#1}}}}
    \newcommand{\NormalTok}[1]{{#1}}
    
    % Additional commands for more recent versions of Pandoc
    \newcommand{\ConstantTok}[1]{\textcolor[rgb]{0.53,0.00,0.00}{{#1}}}
    \newcommand{\SpecialCharTok}[1]{\textcolor[rgb]{0.25,0.44,0.63}{{#1}}}
    \newcommand{\VerbatimStringTok}[1]{\textcolor[rgb]{0.25,0.44,0.63}{{#1}}}
    \newcommand{\SpecialStringTok}[1]{\textcolor[rgb]{0.73,0.40,0.53}{{#1}}}
    \newcommand{\ImportTok}[1]{{#1}}
    \newcommand{\DocumentationTok}[1]{\textcolor[rgb]{0.73,0.13,0.13}{\textit{{#1}}}}
    \newcommand{\AnnotationTok}[1]{\textcolor[rgb]{0.38,0.63,0.69}{\textbf{\textit{{#1}}}}}
    \newcommand{\CommentVarTok}[1]{\textcolor[rgb]{0.38,0.63,0.69}{\textbf{\textit{{#1}}}}}
    \newcommand{\VariableTok}[1]{\textcolor[rgb]{0.10,0.09,0.49}{{#1}}}
    \newcommand{\ControlFlowTok}[1]{\textcolor[rgb]{0.00,0.44,0.13}{\textbf{{#1}}}}
    \newcommand{\OperatorTok}[1]{\textcolor[rgb]{0.40,0.40,0.40}{{#1}}}
    \newcommand{\BuiltInTok}[1]{{#1}}
    \newcommand{\ExtensionTok}[1]{{#1}}
    \newcommand{\PreprocessorTok}[1]{\textcolor[rgb]{0.74,0.48,0.00}{{#1}}}
    \newcommand{\AttributeTok}[1]{\textcolor[rgb]{0.49,0.56,0.16}{{#1}}}
    \newcommand{\InformationTok}[1]{\textcolor[rgb]{0.38,0.63,0.69}{\textbf{\textit{{#1}}}}}
    \newcommand{\WarningTok}[1]{\textcolor[rgb]{0.38,0.63,0.69}{\textbf{\textit{{#1}}}}}
    
    
    % Define a nice break command that doesn't care if a line doesn't already
    % exist.
    \def\br{\hspace*{\fill} \\* }
    % Math Jax compatability definitions
    \def\gt{>}
    \def\lt{<}
    % Document parameters
    \title{Analyse spectrale des séries temporelles de mesures dans l'espace interplanétaire}
    \author{TP A2 - Guillaume \textsc{Bogopolsky}}
    \date{5 octobre 2018} 
    
    

    % Pygments definitions
    
\makeatletter
\def\PY@reset{\let\PY@it=\relax \let\PY@bf=\relax%
    \let\PY@ul=\relax \let\PY@tc=\relax%
    \let\PY@bc=\relax \let\PY@ff=\relax}
\def\PY@tok#1{\csname PY@tok@#1\endcsname}
\def\PY@toks#1+{\ifx\relax#1\empty\else%
    \PY@tok{#1}\expandafter\PY@toks\fi}
\def\PY@do#1{\PY@bc{\PY@tc{\PY@ul{%
    \PY@it{\PY@bf{\PY@ff{#1}}}}}}}
\def\PY#1#2{\PY@reset\PY@toks#1+\relax+\PY@do{#2}}

\expandafter\def\csname PY@tok@w\endcsname{\def\PY@tc##1{\textcolor[rgb]{0.73,0.73,0.73}{##1}}}
\expandafter\def\csname PY@tok@c\endcsname{\let\PY@it=\textit\def\PY@tc##1{\textcolor[rgb]{0.25,0.50,0.50}{##1}}}
\expandafter\def\csname PY@tok@cp\endcsname{\def\PY@tc##1{\textcolor[rgb]{0.74,0.48,0.00}{##1}}}
\expandafter\def\csname PY@tok@k\endcsname{\let\PY@bf=\textbf\def\PY@tc##1{\textcolor[rgb]{0.00,0.50,0.00}{##1}}}
\expandafter\def\csname PY@tok@kp\endcsname{\def\PY@tc##1{\textcolor[rgb]{0.00,0.50,0.00}{##1}}}
\expandafter\def\csname PY@tok@kt\endcsname{\def\PY@tc##1{\textcolor[rgb]{0.69,0.00,0.25}{##1}}}
\expandafter\def\csname PY@tok@o\endcsname{\def\PY@tc##1{\textcolor[rgb]{0.40,0.40,0.40}{##1}}}
\expandafter\def\csname PY@tok@ow\endcsname{\let\PY@bf=\textbf\def\PY@tc##1{\textcolor[rgb]{0.67,0.13,1.00}{##1}}}
\expandafter\def\csname PY@tok@nb\endcsname{\def\PY@tc##1{\textcolor[rgb]{0.00,0.50,0.00}{##1}}}
\expandafter\def\csname PY@tok@nf\endcsname{\def\PY@tc##1{\textcolor[rgb]{0.00,0.00,1.00}{##1}}}
\expandafter\def\csname PY@tok@nc\endcsname{\let\PY@bf=\textbf\def\PY@tc##1{\textcolor[rgb]{0.00,0.00,1.00}{##1}}}
\expandafter\def\csname PY@tok@nn\endcsname{\let\PY@bf=\textbf\def\PY@tc##1{\textcolor[rgb]{0.00,0.00,1.00}{##1}}}
\expandafter\def\csname PY@tok@ne\endcsname{\let\PY@bf=\textbf\def\PY@tc##1{\textcolor[rgb]{0.82,0.25,0.23}{##1}}}
\expandafter\def\csname PY@tok@nv\endcsname{\def\PY@tc##1{\textcolor[rgb]{0.10,0.09,0.49}{##1}}}
\expandafter\def\csname PY@tok@no\endcsname{\def\PY@tc##1{\textcolor[rgb]{0.53,0.00,0.00}{##1}}}
\expandafter\def\csname PY@tok@nl\endcsname{\def\PY@tc##1{\textcolor[rgb]{0.63,0.63,0.00}{##1}}}
\expandafter\def\csname PY@tok@ni\endcsname{\let\PY@bf=\textbf\def\PY@tc##1{\textcolor[rgb]{0.60,0.60,0.60}{##1}}}
\expandafter\def\csname PY@tok@na\endcsname{\def\PY@tc##1{\textcolor[rgb]{0.49,0.56,0.16}{##1}}}
\expandafter\def\csname PY@tok@nt\endcsname{\let\PY@bf=\textbf\def\PY@tc##1{\textcolor[rgb]{0.00,0.50,0.00}{##1}}}
\expandafter\def\csname PY@tok@nd\endcsname{\def\PY@tc##1{\textcolor[rgb]{0.67,0.13,1.00}{##1}}}
\expandafter\def\csname PY@tok@s\endcsname{\def\PY@tc##1{\textcolor[rgb]{0.73,0.13,0.13}{##1}}}
\expandafter\def\csname PY@tok@sd\endcsname{\let\PY@it=\textit\def\PY@tc##1{\textcolor[rgb]{0.73,0.13,0.13}{##1}}}
\expandafter\def\csname PY@tok@si\endcsname{\let\PY@bf=\textbf\def\PY@tc##1{\textcolor[rgb]{0.73,0.40,0.53}{##1}}}
\expandafter\def\csname PY@tok@se\endcsname{\let\PY@bf=\textbf\def\PY@tc##1{\textcolor[rgb]{0.73,0.40,0.13}{##1}}}
\expandafter\def\csname PY@tok@sr\endcsname{\def\PY@tc##1{\textcolor[rgb]{0.73,0.40,0.53}{##1}}}
\expandafter\def\csname PY@tok@ss\endcsname{\def\PY@tc##1{\textcolor[rgb]{0.10,0.09,0.49}{##1}}}
\expandafter\def\csname PY@tok@sx\endcsname{\def\PY@tc##1{\textcolor[rgb]{0.00,0.50,0.00}{##1}}}
\expandafter\def\csname PY@tok@m\endcsname{\def\PY@tc##1{\textcolor[rgb]{0.40,0.40,0.40}{##1}}}
\expandafter\def\csname PY@tok@gh\endcsname{\let\PY@bf=\textbf\def\PY@tc##1{\textcolor[rgb]{0.00,0.00,0.50}{##1}}}
\expandafter\def\csname PY@tok@gu\endcsname{\let\PY@bf=\textbf\def\PY@tc##1{\textcolor[rgb]{0.50,0.00,0.50}{##1}}}
\expandafter\def\csname PY@tok@gd\endcsname{\def\PY@tc##1{\textcolor[rgb]{0.63,0.00,0.00}{##1}}}
\expandafter\def\csname PY@tok@gi\endcsname{\def\PY@tc##1{\textcolor[rgb]{0.00,0.63,0.00}{##1}}}
\expandafter\def\csname PY@tok@gr\endcsname{\def\PY@tc##1{\textcolor[rgb]{1.00,0.00,0.00}{##1}}}
\expandafter\def\csname PY@tok@ge\endcsname{\let\PY@it=\textit}
\expandafter\def\csname PY@tok@gs\endcsname{\let\PY@bf=\textbf}
\expandafter\def\csname PY@tok@gp\endcsname{\let\PY@bf=\textbf\def\PY@tc##1{\textcolor[rgb]{0.00,0.00,0.50}{##1}}}
\expandafter\def\csname PY@tok@go\endcsname{\def\PY@tc##1{\textcolor[rgb]{0.53,0.53,0.53}{##1}}}
\expandafter\def\csname PY@tok@gt\endcsname{\def\PY@tc##1{\textcolor[rgb]{0.00,0.27,0.87}{##1}}}
\expandafter\def\csname PY@tok@err\endcsname{\def\PY@bc##1{\setlength{\fboxsep}{0pt}\fcolorbox[rgb]{1.00,0.00,0.00}{1,1,1}{\strut ##1}}}
\expandafter\def\csname PY@tok@kc\endcsname{\let\PY@bf=\textbf\def\PY@tc##1{\textcolor[rgb]{0.00,0.50,0.00}{##1}}}
\expandafter\def\csname PY@tok@kd\endcsname{\let\PY@bf=\textbf\def\PY@tc##1{\textcolor[rgb]{0.00,0.50,0.00}{##1}}}
\expandafter\def\csname PY@tok@kn\endcsname{\let\PY@bf=\textbf\def\PY@tc##1{\textcolor[rgb]{0.00,0.50,0.00}{##1}}}
\expandafter\def\csname PY@tok@kr\endcsname{\let\PY@bf=\textbf\def\PY@tc##1{\textcolor[rgb]{0.00,0.50,0.00}{##1}}}
\expandafter\def\csname PY@tok@bp\endcsname{\def\PY@tc##1{\textcolor[rgb]{0.00,0.50,0.00}{##1}}}
\expandafter\def\csname PY@tok@fm\endcsname{\def\PY@tc##1{\textcolor[rgb]{0.00,0.00,1.00}{##1}}}
\expandafter\def\csname PY@tok@vc\endcsname{\def\PY@tc##1{\textcolor[rgb]{0.10,0.09,0.49}{##1}}}
\expandafter\def\csname PY@tok@vg\endcsname{\def\PY@tc##1{\textcolor[rgb]{0.10,0.09,0.49}{##1}}}
\expandafter\def\csname PY@tok@vi\endcsname{\def\PY@tc##1{\textcolor[rgb]{0.10,0.09,0.49}{##1}}}
\expandafter\def\csname PY@tok@vm\endcsname{\def\PY@tc##1{\textcolor[rgb]{0.10,0.09,0.49}{##1}}}
\expandafter\def\csname PY@tok@sa\endcsname{\def\PY@tc##1{\textcolor[rgb]{0.73,0.13,0.13}{##1}}}
\expandafter\def\csname PY@tok@sb\endcsname{\def\PY@tc##1{\textcolor[rgb]{0.73,0.13,0.13}{##1}}}
\expandafter\def\csname PY@tok@sc\endcsname{\def\PY@tc##1{\textcolor[rgb]{0.73,0.13,0.13}{##1}}}
\expandafter\def\csname PY@tok@dl\endcsname{\def\PY@tc##1{\textcolor[rgb]{0.73,0.13,0.13}{##1}}}
\expandafter\def\csname PY@tok@s2\endcsname{\def\PY@tc##1{\textcolor[rgb]{0.73,0.13,0.13}{##1}}}
\expandafter\def\csname PY@tok@sh\endcsname{\def\PY@tc##1{\textcolor[rgb]{0.73,0.13,0.13}{##1}}}
\expandafter\def\csname PY@tok@s1\endcsname{\def\PY@tc##1{\textcolor[rgb]{0.73,0.13,0.13}{##1}}}
\expandafter\def\csname PY@tok@mb\endcsname{\def\PY@tc##1{\textcolor[rgb]{0.40,0.40,0.40}{##1}}}
\expandafter\def\csname PY@tok@mf\endcsname{\def\PY@tc##1{\textcolor[rgb]{0.40,0.40,0.40}{##1}}}
\expandafter\def\csname PY@tok@mh\endcsname{\def\PY@tc##1{\textcolor[rgb]{0.40,0.40,0.40}{##1}}}
\expandafter\def\csname PY@tok@mi\endcsname{\def\PY@tc##1{\textcolor[rgb]{0.40,0.40,0.40}{##1}}}
\expandafter\def\csname PY@tok@il\endcsname{\def\PY@tc##1{\textcolor[rgb]{0.40,0.40,0.40}{##1}}}
\expandafter\def\csname PY@tok@mo\endcsname{\def\PY@tc##1{\textcolor[rgb]{0.40,0.40,0.40}{##1}}}
\expandafter\def\csname PY@tok@ch\endcsname{\let\PY@it=\textit\def\PY@tc##1{\textcolor[rgb]{0.25,0.50,0.50}{##1}}}
\expandafter\def\csname PY@tok@cm\endcsname{\let\PY@it=\textit\def\PY@tc##1{\textcolor[rgb]{0.25,0.50,0.50}{##1}}}
\expandafter\def\csname PY@tok@cpf\endcsname{\let\PY@it=\textit\def\PY@tc##1{\textcolor[rgb]{0.25,0.50,0.50}{##1}}}
\expandafter\def\csname PY@tok@c1\endcsname{\let\PY@it=\textit\def\PY@tc##1{\textcolor[rgb]{0.25,0.50,0.50}{##1}}}
\expandafter\def\csname PY@tok@cs\endcsname{\let\PY@it=\textit\def\PY@tc##1{\textcolor[rgb]{0.25,0.50,0.50}{##1}}}

\def\PYZbs{\char`\\}
\def\PYZus{\char`\_}
\def\PYZob{\char`\{}
\def\PYZcb{\char`\}}
\def\PYZca{\char`\^}
\def\PYZam{\char`\&}
\def\PYZlt{\char`\<}
\def\PYZgt{\char`\>}
\def\PYZsh{\char`\#}
\def\PYZpc{\char`\%}
\def\PYZdl{\char`\$}
\def\PYZhy{\char`\-}
\def\PYZsq{\char`\'}
\def\PYZdq{\char`\"}
\def\PYZti{\char`\~}
% for compatibility with earlier versions
\def\PYZat{@}
\def\PYZlb{[}
\def\PYZrb{]}
\makeatother


    % Exact colors from NB
    \definecolor{incolor}{rgb}{0.0, 0.0, 0.5}
    \definecolor{outcolor}{rgb}{0.545, 0.0, 0.0}



    
    % Prevent overflowing lines due to hard-to-break entities
    \sloppy 
    % Setup hyperref package
    \hypersetup{
      breaklinks=true,  % so long urls are correctly broken across lines
      colorlinks=true,
      urlcolor=urlcolor,
      linkcolor=linkcolor,
      citecolor=citecolor,
      }
    % Slightly bigger margins than the latex defaults
    
    \geometry{verbose,tmargin=1in,bmargin=1in,lmargin=1in,rmargin=1in}
    
    

    \begin{document}
    
    
    \maketitle
    
    
    \begin{flushleft}
TP A2 --- Guillaume Bogopolsky\\
Physics of Plasma and Fusion Master's Degree\\
Université Paris-Saclay\\

Document auto-généré grâce à la commande \emph{nbviewer} de \emph{Jupyter}, et retouché manuellement. 
La version HTML est disponible \href{http://nbviewer.jupyter.org/github/gbogopolsky/spectral-analysis-astro/blob/master/Rapport.ipynb?flush_cache=true}{ici}.
    \end{flushleft}

    \hypertarget{introduction}{%
\section{Introduction}\label{introduction}}

    L'analyse spectrale d'un signal consiste à exhiber ses fréquences
caractéristiques grâce à la transformée de Fourier en représentant dans
l'espace des fréquences sa \emph{power spectral density} (PSD).
Toutefois, comme nous le verrons, il est impossible de distinguer la
répartition temporelle des différents modes, contenue dans les phases.
Pour récupérer cette information, nous allons utiliser une transformée
en ondelettes. Cela nous permettra notamment d'étudier des données
spatiales comme les traversées de régions très différentes (ici, la
traversée de la magnétosphère terrestre par la mission CLUSTER).

Au cours de cet TP, nous allons étudier tout d'abord la transformée de
Fourier discrète et la transformée en ondelettes de Morlet sur des
signaux synthétiques, et étudier leurs différences, puis nous
appliquerons ces méthodes à des données réelles.

Ce compte-rendu a été réalisé grâce aux Jupyter Notebook, il vous est
donc possible de récupérer ses fichiers dans le
\href{https://github.com/gbogopolsky/spectral-analysis-astro}{dépôt
GitHub} et de l'éxecuter sur votre machine.

    \hypertarget{eluxe9ments-thuxe9oriques}{%
\section{Eléments théoriques}\label{eluxe9ments-thuxe9oriques}}

Commençons par définir les outils mathématiques que nous allons utiliser
dans cette partie. \#\#\# La transformée de Fourier discrète

    Tout d'abord, définissons la transformée de Fourier discrète (DFT,
\emph{discrete Fourier transform}). Soit une série temporelle
\(u[j] = u(t_{j})\) avec \(t_{j} = j\Delta t = j T/N\) où \(T\) est le
temps d'enregistrement, \(N\) le nombre de mesures et
\(j = 0,1, \ldots ,N-1\) l'indice des points de mesure temporelle. Nous
allons décomposer ce signal sur un nombre fini des fréquences définies
comme \(f_{n} = n/T\) avec \(n = 0,1,\ldots,N - 1\) l'indice des
fréquences.\\
La DFT est alors définie comme suit :
\[ \hat{u}[n] = \frac{1}{N} \sum_{j=0}^{N-1}u[j] e^{-2 i \pi \frac{nj}{N}}\]

La transformée de Fourier inverse, c'est-à-dire l'opération qui à partir
du spectre en fréquence reconstruit le signal temporel, est ainsi :

\[u[j] = \sum_{j=0}^{N-1} \hat{u}[n] e^{2 i \pi \frac{nj}{N}}\]

Une telle transformée a des propriétés intéressantes : pour un signal
réel, il est possible de montrer que :

\[ \hat{u}[N - n] = \hat{u}^*[n]\] \emph{Preuve :}

\[ \hat{u}[N - n] = \frac{1}{N} \sum_{j=0}^{N-1}u[j] e^{-2 i \pi \frac{(N-n)j}{N}} = \frac{1}{N} \sum_{j=0}^{N-1}u[j] e^{-2i \pi j} e^{2 i \pi \frac{nj}{N}} = \frac{1}{N} \sum_{j=0}^{N-1}u[j] e^{2 i \pi \frac{nj}{N}} = \hat{u}^*[n]\]

Nous pouvons aussi démontrer que cette transformation conserve l'énergie
:

\[ \sum_{n=0}^{N-1} \left| \hat{u}[n] \right| ^2  = \sum_{n=0}^{N-1} \hat{u}[n] \hat{u}[n]^* = \frac{1}{N^2} \sum_{n = 0}^{N-1} \left( \sum_{j=0}^{N-1} u[j] e^{-2i\pi \frac{nj}{N}} \right) \left( \sum_{k=0}^{N-1} u[k] e^{2i\pi \frac{nk}{N}} \right)\]
En séparant le produit de sommes en une somme de sommes, il vient :

\[ \sum_{n=0}^{N-1} \left| \hat{u}[n] \right| ^2 = \frac{1}{N^2} \sum_{n = 0}^{N-1} \left( \sum_{\substack{j=0\\ k=0\\ j=k}}^{N-1} u[j] u[k] + \sum_{\substack{j=0\\ k=0\\ j\neq k}}^{N-1} u[j]u[k] e^{2i\pi \frac{n(k-j)}{N}} \right)\]
Inverser les sommations permet d'écrire :

\[ \sum_{n=0}^{N-1} \left| \hat{u}[n] \right| ^2 = \frac{1}{N^2} \sum_{j=0}^{N-1} u[j]^2 \sum_{n=0}^{N-1} 1 + \sum_{\substack{j=0\\ k=0\\ j\neq k}}^{N-1} u[j]u[k] \sum_{n=0}^{N-1} e^{2i\pi \frac{n(k-j)}{N}}\]
où l'on reconnaît la somme des racines de l'unité qui vaut zéro. D'où :

\[ \sum_{n=0}^{N-1} \left| \hat{u}[n] \right| ^2 = \frac{1}{N} \sum_{j=0}^{N-1} \left| u[j]\right|^2\]
C'est le théorème de Parseval-Plancherel.

Nous avons parlé plus haut de la PSD (\emph{power spectral density}).
Elle se définit : \(S[n] = 2T \left| \hat{u}[n] \right|\) avec
\(n = 0,1, \ldots, N/2-1\), car la deuxième moitié du spectre est
identique à la première (comme nous l'avons montré précédemment).
\(S[n]\) s'exprime en \(\text{V}^2 ~\text{Hz}^{-1}\). Nous pouvons
également réécrire le théorème de Plancherel avec la PSD :

\[ \sum_{n=0}^{N-1} S[n] = \frac{2T}{N} \sum_{j=0}^{N-1} \left| u[j]\right|^2\]

    \hypertarget{la-transformuxe9e-en-ondelettes-de-morlet}{%
\subsection{La transformée en ondelettes de
Morlet}\label{la-transformuxe9e-en-ondelettes-de-morlet}}

    Soit l'ondelette de Morlet définit de la façon suivante par rapport au
temps \(t\) : \[ \psi_0 (t) = \pi^{-1/4} e^{-i \omega_0 t} e^{-t^2/2}\]
La dilatation et la translation de cette ondelette permet de définir par
convolution la transformée en ondelettes :
\[\mathcal{W}(\tau, t) = \sum_{j=0}^{N-1} u(t_j) \psi^*((t_j - t)/\tau)\]
avec \(t_j\) la position autour de laquelle on effectue la convolution
dans une fenêtre \(\Delta t\), et \(\tau\) un paramètre de dilatation
temporelle ou échelle temporelle. Nous choisissons cette échelle de
sorte qu'une relation simple existe avec les fréquences de Fourier :
\(1/f \simeq \tau\) en posant \(\omega_0 = 6\).\\
Le carré du module du coefficient d'ondelette
\(\left| \mathcal{W}(\tau, t) \right|^2\) représentant le \emph{quantum}
d'énergie des fluctuations de \(u(t)\) sur la surface
\(\Delta T \times \Delta \tau\) autour d'un moment \(t\) à l'échelle
\(\tau\), nous pouvons tracer le scalogramme de la transformée sur un
graphe à deux dimensions \((t,\tau)\). Il est aussi possible de remonter
à la puissance spectrale équivalente à la PSD via le spectre des
fluctuations en intégrant sur la variable temporelle :

\[S_w = 2 \delta t \mathcal{W}(f)^2 = \frac{2 \delta t}{N} \sum_{i=0}^{N-1} \left| \mathcal{W}(f, t_j) \right|^2\]

Vous trouverez plus de détails
\href{https://sites.lesia.obspm.fr/olga-alexandrova/files/2018/01/TP_M2_PPF_Alexandrova_Cecconi_2017_appendix.pdf}{ici}.

    Maintenant que nos transformées sont définies, nous allons les
implémenter puis les mettre en applications sur des signaux
synthétiques.

\hypertarget{application-uxe0-des-signaux-synthuxe9tiques}{%
\section{ Application à des signaux
synthétiques}\label{application-uxe0-des-signaux-synthuxe9tiques}}

\hypertarget{transformuxe9e-de-fourier}{%
\subsection{Transformée de Fourier}\label{transformuxe9e-de-fourier}}

    Nous implémentons les fonctions dans un fichier annexe
\href{https://github.com/gbogopolsky/spectral-analysis-astro/blob/master/function.py}{\emph{function.py}}.
La DFT est d'abord réalisée de façon intuitive avec des sommes :

\begin{Shaded}
\begin{Highlighting}[]
\KeywordTok{def}\NormalTok{ dft(data):}
    \CommentTok{"""}
\CommentTok{    Discrete Fourier transform of real data.}
\CommentTok{    Input : Time serie of size N}
\CommentTok{    Output : Time serie of size N//2+1}
\CommentTok{    """}
\NormalTok{    N }\OperatorTok{=} \BuiltInTok{len}\NormalTok{(data)}
\NormalTok{    output }\OperatorTok{=}\NormalTok{ np.zeros(N, dtype}\OperatorTok{=}\BuiltInTok{complex}\NormalTok{)}
    \ControlFlowTok{for}\NormalTok{ n }\KeywordTok{in} \BuiltInTok{range}\NormalTok{(N):}
        \ControlFlowTok{for}\NormalTok{ j }\KeywordTok{in} \BuiltInTok{range}\NormalTok{(N):}
\NormalTok{            output[n] }\OperatorTok{+=}\NormalTok{ data[j] }\OperatorTok{*}\NormalTok{ np.exp(}\OperatorTok{-}\DecValTok{2} \OperatorTok{*}\NormalTok{ np.pi }\OperatorTok{*}\NormalTok{ 1j }\OperatorTok{*}\NormalTok{ n }\OperatorTok{*}\NormalTok{ j }\OperatorTok{/}\NormalTok{ N)}
    \ControlFlowTok{return}\NormalTok{ output[:}\BuiltInTok{int}\NormalTok{(N}\OperatorTok{/}\DecValTok{2}\NormalTok{) }\OperatorTok{+} \DecValTok{1}\NormalTok{] }\OperatorTok{/}\NormalTok{ N}
\end{Highlighting}
\end{Shaded}

Une fonction pour générer la liste des fréquences de Fourier est
également pratique :

\begin{Shaded}
\begin{Highlighting}[]
 \KeywordTok{def}\NormalTok{ dftfreq(N, T):}
     \CommentTok{"""}
\CommentTok{     Returns frequency coordinates for the DFT functions.}
\CommentTok{     Input: Size of the time series, and time step.}
\CommentTok{     Output: Frequency serie of size N//2+1}
\CommentTok{     """}
     \ControlFlowTok{return}\NormalTok{ np.arange(N}\OperatorTok{//}\DecValTok{2} \OperatorTok{+} \DecValTok{1}\NormalTok{) }\OperatorTok{/}\NormalTok{ T}
\end{Highlighting}
\end{Shaded}

Appliquons cette fonction aux signaux synthétiques \(y_0(t)\),
\(y_1(t)\) et \(y_2(t)\) définis également dans le même fichier :

    \begin{Verbatim}[commandchars=\\\{\}]
{\color{incolor}In [{\color{incolor}2}]:} \PY{k+kn}{import} \PY{n+nn}{numpy} \PY{k}{as} \PY{n+nn}{np}
        \PY{k+kn}{import} \PY{n+nn}{matplotlib}\PY{n+nn}{.}\PY{n+nn}{pyplot} \PY{k}{as} \PY{n+nn}{plt}
        \PY{k+kn}{import} \PY{n+nn}{function} \PY{k}{as} \PY{n+nn}{fct}
        \PY{n}{plt}\PY{o}{.}\PY{n}{rcParams}\PY{p}{[}\PY{l+s+s1}{\PYZsq{}}\PY{l+s+s1}{figure.figsize}\PY{l+s+s1}{\PYZsq{}}\PY{p}{]} \PY{o}{=} \PY{p}{(}\PY{l+m+mi}{17}\PY{p}{,}\PY{l+m+mi}{10}\PY{p}{)}
        \PY{n}{plt}\PY{o}{.}\PY{n}{rcParams}\PY{p}{[}\PY{l+s+s1}{\PYZsq{}}\PY{l+s+s1}{font.size}\PY{l+s+s1}{\PYZsq{}}\PY{p}{]} \PY{o}{=} \PY{l+m+mi}{17}
        \PY{c+c1}{\PYZsh{} Constants}
        \PY{n}{N} \PY{o}{=} \PY{n+nb}{int}\PY{p}{(}\PY{l+m+mf}{1e3}\PY{p}{)}
        \PY{n}{T} \PY{o}{=} \PY{l+m+mi}{10}
        \PY{n}{dt} \PY{o}{=} \PY{n}{T}\PY{o}{/}\PY{n}{N}
        \PY{n}{A0}\PY{p}{,} \PY{n}{f0} \PY{o}{=} \PY{l+m+mi}{10}\PY{p}{,} \PY{l+m+mf}{0.5}
        \PY{n}{A1}\PY{p}{,} \PY{n}{f1} \PY{o}{=} \PY{l+m+mi}{8}\PY{p}{,} \PY{l+m+mi}{1}
\end{Verbatim}


    \begin{Verbatim}[commandchars=\\\{\}]
{\color{incolor}In [{\color{incolor}3}]:} \PY{n}{t} \PY{o}{=} \PY{n}{np}\PY{o}{.}\PY{n}{arange}\PY{p}{(}\PY{n}{N}\PY{p}{)} \PY{o}{*} \PY{n}{dt}
        \PY{n}{y0}\PY{p}{,} \PY{n}{y1}\PY{p}{,} \PY{n}{y2} \PY{o}{=} \PY{n}{fct}\PY{o}{.}\PY{n}{y0}\PY{p}{(}\PY{n}{t}\PY{p}{,} \PY{n}{A0}\PY{p}{,} \PY{n}{f0}\PY{p}{)}\PY{p}{,} \PY{n}{fct}\PY{o}{.}\PY{n}{y1}\PY{p}{(}\PY{n}{t}\PY{p}{,} \PY{n}{A0}\PY{p}{,} \PY{n}{A1}\PY{p}{,} \PY{n}{f0}\PY{p}{,} \PY{n}{f1}\PY{p}{)}\PY{p}{,} \PY{n}{fct}\PY{o}{.}\PY{n}{y2}\PY{p}{(}\PY{n}{t}\PY{p}{,} \PY{n}{A0}\PY{p}{,} \PY{n}{A1}\PY{p}{,} \PY{n}{f0}\PY{p}{,} \PY{n}{f1}\PY{p}{,} \PY{n}{T}\PY{p}{)}
        \PY{n}{PSD0} \PY{o}{=} \PY{l+m+mi}{2} \PY{o}{*} \PY{n}{T} \PY{o}{*} \PY{n}{np}\PY{o}{.}\PY{n}{abs}\PY{p}{(}\PY{n}{fct}\PY{o}{.}\PY{n}{fdft}\PY{p}{(}\PY{n}{y0}\PY{p}{)}\PY{p}{)}\PY{o}{*}\PY{o}{*}\PY{l+m+mi}{2}
        \PY{n}{PSD1} \PY{o}{=} \PY{l+m+mi}{2} \PY{o}{*} \PY{n}{T} \PY{o}{*} \PY{n}{np}\PY{o}{.}\PY{n}{abs}\PY{p}{(}\PY{n}{fct}\PY{o}{.}\PY{n}{fdft}\PY{p}{(}\PY{n}{y1}\PY{p}{)}\PY{p}{)}\PY{o}{*}\PY{o}{*}\PY{l+m+mi}{2}
        \PY{n}{PSD2} \PY{o}{=} \PY{l+m+mi}{2} \PY{o}{*} \PY{n}{T} \PY{o}{*} \PY{n}{np}\PY{o}{.}\PY{n}{abs}\PY{p}{(}\PY{n}{fct}\PY{o}{.}\PY{n}{fdft}\PY{p}{(}\PY{n}{y2}\PY{p}{)}\PY{p}{)}\PY{o}{*}\PY{o}{*}\PY{l+m+mi}{2}
        \PY{n}{freq} \PY{o}{=} \PY{n}{fct}\PY{o}{.}\PY{n}{dftfreq}\PY{p}{(}\PY{n}{N}\PY{p}{,} \PY{n}{T}\PY{p}{)}
\end{Verbatim}


    \begin{Verbatim}[commandchars=\\\{\}]
{\color{incolor}In [{\color{incolor}5}]:} \PY{n}{plt}\PY{o}{.}\PY{n}{subplot}\PY{p}{(}\PY{l+m+mi}{211}\PY{p}{)}
        \PY{n}{plt}\PY{o}{.}\PY{n}{plot}\PY{p}{(}\PY{n}{t}\PY{p}{,} \PY{n}{y0}\PY{p}{,} \PY{n}{linewidth}\PY{o}{=}\PY{l+m+mf}{1.2}\PY{p}{,} \PY{n}{label}\PY{o}{=}\PY{l+s+sa}{r}\PY{l+s+s1}{\PYZsq{}}\PY{l+s+s1}{\PYZdl{}y\PYZus{}0\PYZdl{}}\PY{l+s+s1}{\PYZsq{}}\PY{p}{)}
        \PY{n}{plt}\PY{o}{.}\PY{n}{plot}\PY{p}{(}\PY{n}{t}\PY{p}{,} \PY{n}{y1}\PY{p}{,} \PY{n}{linewidth}\PY{o}{=}\PY{l+m+mi}{1}\PY{p}{,} \PY{n}{label}\PY{o}{=}\PY{l+s+sa}{r}\PY{l+s+s1}{\PYZsq{}}\PY{l+s+s1}{\PYZdl{}y\PYZus{}1\PYZdl{}}\PY{l+s+s1}{\PYZsq{}}\PY{p}{)}
        \PY{n}{plt}\PY{o}{.}\PY{n}{plot}\PY{p}{(}\PY{n}{t}\PY{p}{,} \PY{n}{y2}\PY{p}{,} \PY{n}{linewidth}\PY{o}{=}\PY{l+m+mi}{1}\PY{p}{,} \PY{n}{label}\PY{o}{=}\PY{l+s+sa}{r}\PY{l+s+s1}{\PYZsq{}}\PY{l+s+s1}{\PYZdl{}y\PYZus{}2\PYZdl{}}\PY{l+s+s1}{\PYZsq{}}\PY{p}{)}
        \PY{n}{plt}\PY{o}{.}\PY{n}{xlabel}\PY{p}{(}\PY{l+s+s1}{\PYZsq{}}\PY{l+s+s1}{Time (s)}\PY{l+s+s1}{\PYZsq{}}\PY{p}{)}
        \PY{n}{plt}\PY{o}{.}\PY{n}{ylabel}\PY{p}{(}\PY{l+s+s1}{\PYZsq{}}\PY{l+s+s1}{y(t) (V)}\PY{l+s+s1}{\PYZsq{}}\PY{p}{)}
        \PY{n}{plt}\PY{o}{.}\PY{n}{xlim}\PY{p}{(}\PY{p}{(}\PY{n}{np}\PY{o}{.}\PY{n}{min}\PY{p}{(}\PY{n}{t}\PY{p}{)}\PY{p}{,} \PY{n}{np}\PY{o}{.}\PY{n}{max}\PY{p}{(}\PY{n}{t}\PY{p}{)}\PY{p}{)}\PY{p}{)}
        \PY{n}{plt}\PY{o}{.}\PY{n}{legend}\PY{p}{(}\PY{p}{)}
        
        \PY{n}{plt}\PY{o}{.}\PY{n}{subplot}\PY{p}{(}\PY{l+m+mi}{212}\PY{p}{)}
        \PY{n}{plt}\PY{o}{.}\PY{n}{semilogx}\PY{p}{(}\PY{n}{freq}\PY{p}{,} \PY{n}{PSD0}\PY{p}{,} \PY{n}{linewidth}\PY{o}{=}\PY{l+m+mi}{1}\PY{p}{,} \PY{n}{label}\PY{o}{=}\PY{l+s+sa}{r}\PY{l+s+s1}{\PYZsq{}}\PY{l+s+s1}{\PYZdl{}}\PY{l+s+s1}{\PYZbs{}}\PY{l+s+s1}{hat}\PY{l+s+si}{\PYZob{}y\PYZcb{}}\PY{l+s+s1}{\PYZus{}0\PYZdl{}}\PY{l+s+s1}{\PYZsq{}}\PY{p}{)}
        \PY{n}{plt}\PY{o}{.}\PY{n}{semilogx}\PY{p}{(}\PY{n}{freq}\PY{p}{,} \PY{n}{PSD1}\PY{p}{,} \PY{n}{linewidth}\PY{o}{=}\PY{l+m+mi}{1}\PY{p}{,} \PY{n}{label}\PY{o}{=}\PY{l+s+sa}{r}\PY{l+s+s1}{\PYZsq{}}\PY{l+s+s1}{\PYZdl{}}\PY{l+s+s1}{\PYZbs{}}\PY{l+s+s1}{hat}\PY{l+s+si}{\PYZob{}y\PYZcb{}}\PY{l+s+s1}{\PYZus{}1\PYZdl{}}\PY{l+s+s1}{\PYZsq{}}\PY{p}{)}
        \PY{n}{plt}\PY{o}{.}\PY{n}{semilogx}\PY{p}{(}\PY{n}{freq}\PY{p}{,} \PY{n}{PSD2}\PY{p}{,} \PY{n}{linewidth}\PY{o}{=}\PY{l+m+mi}{1}\PY{p}{,} \PY{n}{label}\PY{o}{=}\PY{l+s+sa}{r}\PY{l+s+s1}{\PYZsq{}}\PY{l+s+s1}{\PYZdl{}}\PY{l+s+s1}{\PYZbs{}}\PY{l+s+s1}{hat}\PY{l+s+si}{\PYZob{}y\PYZcb{}}\PY{l+s+s1}{\PYZus{}2\PYZdl{}}\PY{l+s+s1}{\PYZsq{}}\PY{p}{)}
        \PY{n}{plt}\PY{o}{.}\PY{n}{vlines}\PY{p}{(}\PY{p}{(}\PY{n}{f0}\PY{p}{,} \PY{n}{f1}\PY{p}{)}\PY{p}{,} \PY{l+m+mi}{0}\PY{p}{,} \PY{l+m+mi}{500}\PY{p}{,} \PY{n}{linewidth}\PY{o}{=}\PY{l+m+mi}{1}\PY{p}{,} \PY{n}{color}\PY{o}{=}\PY{l+s+s1}{\PYZsq{}}\PY{l+s+s1}{blue}\PY{l+s+s1}{\PYZsq{}}\PY{p}{)}
        \PY{n}{plt}\PY{o}{.}\PY{n}{xlim}\PY{p}{(}\PY{p}{(}\PY{n}{np}\PY{o}{.}\PY{n}{abs}\PY{p}{(}\PY{n}{np}\PY{o}{.}\PY{n}{min}\PY{p}{(}\PY{n}{freq}\PY{p}{)}\PY{p}{)}\PY{p}{,} \PY{n}{np}\PY{o}{.}\PY{n}{max}\PY{p}{(}\PY{n}{freq}\PY{p}{)}\PY{p}{)}\PY{p}{)}
        \PY{n}{plt}\PY{o}{.}\PY{n}{xlabel}\PY{p}{(}\PY{l+s+s1}{\PYZsq{}}\PY{l+s+s1}{Frequency (Hz)}\PY{l+s+s1}{\PYZsq{}}\PY{p}{)}
        \PY{n}{plt}\PY{o}{.}\PY{n}{ylabel}\PY{p}{(}\PY{l+s+s1}{\PYZsq{}}\PY{l+s+s1}{PSD (V\PYZca{}2 / Hz)}\PY{l+s+s1}{\PYZsq{}}\PY{p}{)}
        \PY{n}{plt}\PY{o}{.}\PY{n}{legend}\PY{p}{(}\PY{p}{)}
        \PY{n}{plt}\PY{o}{.}\PY{n}{tight\PYZus{}layout}\PY{p}{(}\PY{p}{)}
\end{Verbatim}


    \begin{center}
    \adjustimage{max size={0.9\linewidth}{0.9\paperheight}}{Rapport_files/Rapport_11_0.png}
    \end{center}
    { \hspace*{\fill} \\}
    
    Nous constatons tout d'abord que seule la moitié de la gamme de
fréquence est couverte : en effet, la symétrie $ \hat{u}\left[N - n\right] =
\hat{u}^{*}\left[n\right]$ nous confirme que ces deux moitiés sont identiques
pour des données réelles. Il n'est donc pas nécessaire de conserver le
reste. De plus, la fréquence maximale couverte est la fréquence de
Nyquist, donnée par le théorème de Shannon-Nyquist :
\(f_{max} = f_{echant}/2 = 1/(2dt)\).\\
Sur les spectres, les fréquences \(f_0\) et \(f_1\) sont bien visibles,
mais une chose est très claire : en ne se basant que sur la transformée
de Fourier, il est impossible de distinguer les signaux \(y_1\) et
\(y_2\), qui diffèrent par le déphasage entre les modes associés à ces
deux fréquences. Cela sera possible avec transformée en ondelettes.

    Nous avons également confirmé la précision de notre DFT à l'aide des
transformées implémentées dans \texttt{numpy.fft} (cf.
\href{http://nbviewer.jupyter.org/github/gbogopolsky/spectral-analysis-astro/blob/master/TPA2-donnees-synthetiques.ipynb?flush_cache=true}{TPA2-donnees-synthetiques.ipynb}),
avec succès.\\
Par ailleurs, vous avez peut-être remarqué précédemment l'utilisation de
la fonction \texttt{fdft} au lieu de \texttt{dft}. En effet, la DFT
implémentée comme nous l'avons décrit est très lente à cause de
l'utilisation des doubles boucles de Python. Afin d'optimiser la vitesse
d'exécution, nous mettons à profit la vitesse des opérations
matricielles codées en C du paquet Numpy en implémentant \texttt{fdft}
de la manière suivante :

\begin{Shaded}
\begin{Highlighting}[]
\KeywordTok{def}\NormalTok{ fdft(data):}
    \CommentTok{"""}
\CommentTok{    Fast discrete Fourier transform of real data with Numpy and matrix operations.}
\CommentTok{    Input: Time serie of size N}
\CommentTok{    Output: Time serie of size N//2+1}
\CommentTok{    """}
\NormalTok{    N }\OperatorTok{=} \BuiltInTok{len}\NormalTok{(data)}
\NormalTok{    exp }\OperatorTok{=}\NormalTok{ np.exp(}\OperatorTok{-}\DecValTok{2} \OperatorTok{*}\NormalTok{ np.pi }\OperatorTok{*}\NormalTok{ 1j }\OperatorTok{/}\NormalTok{ N }\OperatorTok{*}\NormalTok{ np.dot(np.arange(N)[:,np.newaxis], np.arange(N)[np.newaxis,:]))}
\NormalTok{    output }\OperatorTok{=}\NormalTok{ np.dot(data, exp) }\OperatorTok{/}\NormalTok{ N}
    \ControlFlowTok{return}\NormalTok{ output[:N}\OperatorTok{//}\DecValTok{2}\OperatorTok{+}\DecValTok{1}\NormalTok{]}
\end{Highlighting}
\end{Shaded}

Dans cette version, nous créons une matrice contenant tous les termes de
l'exponentielle complexe en se basant sur la matrice \(M_{n,j} = n j\)
créée par produit matriciel, les axes étant choisi de façon adéquate. Il
suffit ensuite de la multiplier avec notre vecteur de données pour
obtenir la série fréquentielle. L'optimisation n'est pas encore optimale
(la sortie étant tronquée, la routine calcule deux fois trop de
valeurs), mais elle nous permet tout de même d'augmenter la vitesse
d'exécution un facteur 48 !

    \begin{Verbatim}[commandchars=\\\{\}]
{\color{incolor}In [{\color{incolor}6}]:} \PY{o}{\PYZpc{}\PYZpc{}}\PY{k}{timeit}
        test = 2*T*np.abs(fct.dft(y0))**2
\end{Verbatim}


    \begin{Verbatim}[commandchars=\\\{\}]
3.97 s ± 148 ms per loop (mean ± std. dev. of 7 runs, 1 loop each)

    \end{Verbatim}

    \begin{Verbatim}[commandchars=\\\{\}]
{\color{incolor}In [{\color{incolor}7}]:} \PY{o}{\PYZpc{}\PYZpc{}}\PY{k}{timeit}
        test = 2*T*np.abs(fct.fdft(y0))**2
\end{Verbatim}


    \begin{Verbatim}[commandchars=\\\{\}]
81.4 ms ± 579 µs per loop (mean ± std. dev. of 7 runs, 10 loops each)

    \end{Verbatim}

    \hypertarget{transformuxe9e-en-ondelettes-de-morlet}{%
\subsection{Transformée en ondelettes de
Morlet}\label{transformuxe9e-en-ondelettes-de-morlet}}

    Afin de réaliser la transformée en ondelettes, nous allons utiliser une
routine développée par C. Torrence, G. Compo pour leur article et
adaptée en Python par E. Predybaylo, trouvable sur leur
\href{https://github.com/chris-torrence/wavelets}{dépôt}.

    \begin{Verbatim}[commandchars=\\\{\}]
{\color{incolor}In [{\color{incolor}8}]:} \PY{k+kn}{import} \PY{n+nn}{waveletFunctions} \PY{k}{as} \PY{n+nn}{wav}
\end{Verbatim}


    Nous nous intéressons aux transformées des signaux \(y_1\) et \(y_2\),
indiscernables par simple transformée de Fourier.

    \begin{Verbatim}[commandchars=\\\{\}]
{\color{incolor}In [{\color{incolor}9}]:} \PY{n}{wave1}\PY{p}{,} \PY{n}{period1}\PY{p}{,} \PY{n}{scale1}\PY{p}{,} \PY{n}{coi1} \PY{o}{=} \PY{n}{wav}\PY{o}{.}\PY{n}{wavelet}\PY{p}{(}\PY{n}{y1}\PY{p}{,} \PY{n}{dt}\PY{p}{)}
        \PY{n}{wave2}\PY{p}{,} \PY{n}{period2}\PY{p}{,} \PY{n}{scale2}\PY{p}{,} \PY{n}{coi2} \PY{o}{=} \PY{n}{wav}\PY{o}{.}\PY{n}{wavelet}\PY{p}{(}\PY{n}{y2}\PY{p}{,} \PY{n}{dt}\PY{p}{)}
\end{Verbatim}


    \begin{Verbatim}[commandchars=\\\{\}]
{\color{incolor}In [{\color{incolor}10}]:} \PY{n}{plt}\PY{o}{.}\PY{n}{subplot}\PY{p}{(}\PY{l+m+mi}{211}\PY{p}{)}
         \PY{n}{contour\PYZus{}plot} \PY{o}{=} \PY{n}{plt}\PY{o}{.}\PY{n}{contourf}\PY{p}{(}\PY{n}{t}\PY{p}{,} \PY{n}{period1}\PY{p}{,} \PY{n}{np}\PY{o}{.}\PY{n}{abs}\PY{p}{(}\PY{n}{wave1}\PY{p}{)}\PY{o}{*}\PY{o}{*}\PY{l+m+mi}{2}\PY{p}{,} \PY{l+m+mi}{100}\PY{p}{,} \PY{n}{extend}\PY{o}{=}\PY{l+s+s1}{\PYZsq{}}\PY{l+s+s1}{both}\PY{l+s+s1}{\PYZsq{}}\PY{p}{)}
         \PY{n}{plt}\PY{o}{.}\PY{n}{xlabel}\PY{p}{(}\PY{l+s+s1}{\PYZsq{}}\PY{l+s+s1}{Time (s)}\PY{l+s+s1}{\PYZsq{}}\PY{p}{)}
         \PY{n}{plt}\PY{o}{.}\PY{n}{ylabel}\PY{p}{(}\PY{l+s+sa}{r}\PY{l+s+s1}{\PYZsq{}}\PY{l+s+s1}{Period (s)}\PY{l+s+s1}{\PYZsq{}}\PY{p}{)}
         \PY{n}{plt}\PY{o}{.}\PY{n}{title}\PY{p}{(}\PY{l+s+sa}{r}\PY{l+s+s1}{\PYZsq{}}\PY{l+s+s1}{Scaleogram of \PYZdl{}y\PYZus{}}\PY{l+s+si}{\PYZob{}1\PYZcb{}}\PY{l+s+s1}{\PYZdl{}}\PY{l+s+s1}{\PYZsq{}}\PY{p}{)}
         \PY{n}{cbar} \PY{o}{=} \PY{n}{plt}\PY{o}{.}\PY{n}{colorbar}\PY{p}{(}\PY{n}{contour\PYZus{}plot}\PY{p}{)}
         \PY{n}{cbar}\PY{o}{.}\PY{n}{set\PYZus{}label}\PY{p}{(}\PY{l+s+sa}{r}\PY{l+s+s1}{\PYZsq{}}\PY{l+s+s1}{\PYZdl{}}\PY{l+s+s1}{\PYZbs{}}\PY{l+s+s1}{left| }\PY{l+s+s1}{\PYZbs{}}\PY{l+s+s1}{mathcal}\PY{l+s+si}{\PYZob{}W\PYZcb{}}\PY{l+s+s1}{(}\PY{l+s+s1}{\PYZbs{}}\PY{l+s+s1}{tau,t)}\PY{l+s+s1}{\PYZbs{}}\PY{l+s+s1}{right| \PYZca{}}\PY{l+s+si}{\PYZob{}2\PYZcb{}}\PY{l+s+s1}{\PYZdl{}}\PY{l+s+s1}{\PYZsq{}}\PY{p}{,} \PY{n}{rotation}\PY{o}{=}\PY{l+m+mi}{270}\PY{p}{,} \PY{n}{labelpad}\PY{o}{=}\PY{l+m+mi}{30}\PY{p}{)}
         
         \PY{n}{plt}\PY{o}{.}\PY{n}{subplot}\PY{p}{(}\PY{l+m+mi}{212}\PY{p}{)}
         \PY{n}{contour\PYZus{}plot} \PY{o}{=} \PY{n}{plt}\PY{o}{.}\PY{n}{contourf}\PY{p}{(}\PY{n}{t}\PY{p}{,} \PY{n}{period2}\PY{p}{,} \PY{n}{np}\PY{o}{.}\PY{n}{abs}\PY{p}{(}\PY{n}{wave2}\PY{p}{)}\PY{o}{*}\PY{o}{*}\PY{l+m+mi}{2}\PY{p}{,} \PY{l+m+mi}{100}\PY{p}{,} \PY{n}{extend}\PY{o}{=}\PY{l+s+s1}{\PYZsq{}}\PY{l+s+s1}{both}\PY{l+s+s1}{\PYZsq{}}\PY{p}{)}
         \PY{n}{plt}\PY{o}{.}\PY{n}{xlabel}\PY{p}{(}\PY{l+s+s1}{\PYZsq{}}\PY{l+s+s1}{Time (s)}\PY{l+s+s1}{\PYZsq{}}\PY{p}{)}
         \PY{n}{plt}\PY{o}{.}\PY{n}{ylabel}\PY{p}{(}\PY{l+s+sa}{r}\PY{l+s+s1}{\PYZsq{}}\PY{l+s+s1}{Period (s)}\PY{l+s+s1}{\PYZsq{}}\PY{p}{)}
         \PY{n}{plt}\PY{o}{.}\PY{n}{title}\PY{p}{(}\PY{l+s+sa}{r}\PY{l+s+s1}{\PYZsq{}}\PY{l+s+s1}{Scaleogram of \PYZdl{}y\PYZus{}}\PY{l+s+si}{\PYZob{}2\PYZcb{}}\PY{l+s+s1}{\PYZdl{}}\PY{l+s+s1}{\PYZsq{}}\PY{p}{)}
         \PY{n}{cbar} \PY{o}{=} \PY{n}{plt}\PY{o}{.}\PY{n}{colorbar}\PY{p}{(}\PY{n}{contour\PYZus{}plot}\PY{p}{)}
         \PY{n}{cbar}\PY{o}{.}\PY{n}{set\PYZus{}label}\PY{p}{(}\PY{l+s+sa}{r}\PY{l+s+s1}{\PYZsq{}}\PY{l+s+s1}{\PYZdl{}}\PY{l+s+s1}{\PYZbs{}}\PY{l+s+s1}{left| }\PY{l+s+s1}{\PYZbs{}}\PY{l+s+s1}{mathcal}\PY{l+s+si}{\PYZob{}W\PYZcb{}}\PY{l+s+s1}{(}\PY{l+s+s1}{\PYZbs{}}\PY{l+s+s1}{tau,t)}\PY{l+s+s1}{\PYZbs{}}\PY{l+s+s1}{right| \PYZca{}}\PY{l+s+si}{\PYZob{}2\PYZcb{}}\PY{l+s+s1}{\PYZdl{}}\PY{l+s+s1}{\PYZsq{}}\PY{p}{,} \PY{n}{rotation}\PY{o}{=}\PY{l+m+mi}{270}\PY{p}{,} \PY{n}{labelpad}\PY{o}{=}\PY{l+m+mi}{30}\PY{p}{)}
         \PY{n}{plt}\PY{o}{.}\PY{n}{tight\PYZus{}layout}\PY{p}{(}\PY{p}{)}
\end{Verbatim}


    \begin{center}
    \adjustimage{max size={0.9\linewidth}{0.9\paperheight}}{Rapport_files/Rapport_21_0.png}
    \end{center}
    { \hspace*{\fill} \\}
    
    Nous remarquons clairement la différence entre les deux signaux : dans
le premier cas, les deux modes \(f_0\) et \(f_1\) existent sur toute la
durée du signal, tandis que dans le deuxième cas, le mode \(f_1 = 1,0\)
Hz succède au mode \(f_0 = 0,5\) Hz à partir de \(T/2 = 5\) s.

    \begin{Verbatim}[commandchars=\\\{\}]
{\color{incolor}In [{\color{incolor}11}]:} \PY{n+nb}{print}\PY{p}{(}\PY{l+s+s1}{\PYZsq{}}\PY{l+s+s1}{T0 = }\PY{l+s+s1}{\PYZsq{}}\PY{p}{,} \PY{l+m+mi}{1}\PY{o}{/}\PY{n}{f0}\PY{p}{,} \PY{l+s+s1}{\PYZsq{}}\PY{l+s+s1}{ T1 = }\PY{l+s+s1}{\PYZsq{}}\PY{p}{,} \PY{l+m+mi}{1}\PY{o}{/}\PY{n}{f1}\PY{p}{)}
\end{Verbatim}


    \begin{Verbatim}[commandchars=\\\{\}]
T0 =  2.0  T1 =  1.0

    \end{Verbatim}

    \hypertarget{comparaison-entre-les-deux-muxe9thodes}{%
\subsection{Comparaison entre les deux
méthodes}\label{comparaison-entre-les-deux-muxe9thodes}}

    Enfin, comparons pour le signal \(y_1\) le spectre de Fourier et le
spectre des wavelets obtenu est calculant la grandeur \(S_w\) définie
précédemment :

    \begin{Verbatim}[commandchars=\\\{\}]
{\color{incolor}In [{\color{incolor}12}]:} \PY{n}{spWy1} \PY{o}{=} \PY{l+m+mi}{2}\PY{o}{*}\PY{n}{dt}\PY{o}{*}\PY{n}{np}\PY{o}{.}\PY{n}{sum}\PY{p}{(}\PY{n}{np}\PY{o}{.}\PY{n}{abs}\PY{p}{(}\PY{n}{wave1}\PY{p}{)}\PY{o}{*}\PY{o}{*}\PY{l+m+mi}{2}\PY{p}{,} \PY{n}{axis}\PY{o}{=}\PY{l+m+mi}{1}\PY{p}{)}\PY{o}{/}\PY{n}{wave1}\PY{o}{.}\PY{n}{shape}\PY{p}{[}\PY{l+m+mi}{1}\PY{p}{]}
\end{Verbatim}


    \begin{Verbatim}[commandchars=\\\{\}]
{\color{incolor}In [{\color{incolor}14}]:} \PY{n}{plt}\PY{o}{.}\PY{n}{subplot}\PY{p}{(}\PY{l+m+mi}{211}\PY{p}{)}
         \PY{n}{plt}\PY{o}{.}\PY{n}{semilogx}\PY{p}{(}\PY{n}{freq}\PY{p}{,} \PY{n}{PSD1}\PY{p}{,} \PY{n}{linewidth}\PY{o}{=}\PY{l+m+mf}{1.2}\PY{p}{,} \PY{n}{color}\PY{o}{=}\PY{l+s+s1}{\PYZsq{}}\PY{l+s+s1}{red}\PY{l+s+s1}{\PYZsq{}}\PY{p}{)}
         \PY{n}{plt}\PY{o}{.}\PY{n}{xlim}\PY{p}{(}\PY{p}{(}\PY{n}{np}\PY{o}{.}\PY{n}{abs}\PY{p}{(}\PY{n}{np}\PY{o}{.}\PY{n}{min}\PY{p}{(}\PY{n}{freq}\PY{p}{)}\PY{p}{)}\PY{p}{,} \PY{n}{np}\PY{o}{.}\PY{n}{max}\PY{p}{(}\PY{n}{freq}\PY{p}{)}\PY{p}{)}\PY{p}{)}
         \PY{n}{plt}\PY{o}{.}\PY{n}{vlines}\PY{p}{(}\PY{p}{[}\PY{n}{f0}\PY{p}{,} \PY{n}{f1}\PY{p}{]}\PY{p}{,} \PY{n}{np}\PY{o}{.}\PY{n}{min}\PY{p}{(}\PY{n}{PSD1}\PY{p}{)}\PY{p}{,} \PY{n}{np}\PY{o}{.}\PY{n}{max}\PY{p}{(}\PY{n}{PSD1}\PY{p}{)}\PY{p}{,} \PY{n}{color}\PY{o}{=}\PY{l+s+s1}{\PYZsq{}}\PY{l+s+s1}{blue}\PY{l+s+s1}{\PYZsq{}}\PY{p}{,} \PY{n}{linewidth}\PY{o}{=}\PY{l+m+mi}{1}\PY{p}{)}
         \PY{n}{plt}\PY{o}{.}\PY{n}{xlabel}\PY{p}{(}\PY{l+s+s1}{\PYZsq{}}\PY{l+s+s1}{Frequency (Hz)}\PY{l+s+s1}{\PYZsq{}}\PY{p}{)}
         \PY{n}{plt}\PY{o}{.}\PY{n}{ylabel}\PY{p}{(}\PY{l+s+s1}{\PYZsq{}}\PY{l+s+s1}{PSD (V\PYZca{}2 / Hz)}\PY{l+s+s1}{\PYZsq{}}\PY{p}{)}
         
         \PY{n}{plt}\PY{o}{.}\PY{n}{subplot}\PY{p}{(}\PY{l+m+mi}{212}\PY{p}{)}
         \PY{n}{plt}\PY{o}{.}\PY{n}{semilogx}\PY{p}{(}\PY{l+m+mi}{1}\PY{o}{/}\PY{n}{period1}\PY{p}{,} \PY{n}{spWy1}\PY{p}{,} \PY{n}{linewidth}\PY{o}{=}\PY{l+m+mf}{1.2}\PY{p}{,} \PY{n}{color}\PY{o}{=}\PY{l+s+s1}{\PYZsq{}}\PY{l+s+s1}{red}\PY{l+s+s1}{\PYZsq{}}\PY{p}{)}
         \PY{n}{plt}\PY{o}{.}\PY{n}{vlines}\PY{p}{(}\PY{p}{[}\PY{n}{f0}\PY{p}{,} \PY{n}{f1}\PY{p}{]}\PY{p}{,} \PY{n}{np}\PY{o}{.}\PY{n}{min}\PY{p}{(}\PY{n}{spWy1}\PY{p}{)}\PY{p}{,} \PY{n}{np}\PY{o}{.}\PY{n}{max}\PY{p}{(}\PY{n}{spWy1}\PY{p}{)}\PY{p}{,} \PY{n}{color}\PY{o}{=}\PY{l+s+s1}{\PYZsq{}}\PY{l+s+s1}{blue}\PY{l+s+s1}{\PYZsq{}}\PY{p}{,} \PY{n}{linewidth}\PY{o}{=}\PY{l+m+mi}{1}\PY{p}{)}
         \PY{n}{plt}\PY{o}{.}\PY{n}{xlim}\PY{p}{(}\PY{n}{np}\PY{o}{.}\PY{n}{abs}\PY{p}{(}\PY{n}{np}\PY{o}{.}\PY{n}{min}\PY{p}{(}\PY{l+m+mi}{1}\PY{o}{/}\PY{n}{period1}\PY{p}{)}\PY{p}{)}\PY{p}{,} \PY{n}{np}\PY{o}{.}\PY{n}{max}\PY{p}{(}\PY{l+m+mi}{1}\PY{o}{/}\PY{n}{period1}\PY{p}{)}\PY{p}{)}
         \PY{n}{plt}\PY{o}{.}\PY{n}{xlabel}\PY{p}{(}\PY{l+s+s1}{\PYZsq{}}\PY{l+s+s1}{Frequency (Hz)}\PY{l+s+s1}{\PYZsq{}}\PY{p}{)}
         \PY{n}{plt}\PY{o}{.}\PY{n}{ylabel}\PY{p}{(}\PY{l+s+sa}{r}\PY{l+s+s1}{\PYZsq{}}\PY{l+s+s1}{\PYZdl{} S\PYZus{}}\PY{l+s+si}{\PYZob{}w\PYZcb{}}\PY{l+s+s1}{(f)\PYZdl{} (V\PYZca{}2 / Hz)}\PY{l+s+s1}{\PYZsq{}}\PY{p}{)}
         \PY{n}{plt}\PY{o}{.}\PY{n}{tight\PYZus{}layout}\PY{p}{(}\PY{p}{)}
\end{Verbatim}


    \begin{center}
    \adjustimage{max size={0.9\linewidth}{0.9\paperheight}}{Rapport_files/Rapport_27_0.png}
    \end{center}
    { \hspace*{\fill} \\}
    
    \begin{Verbatim}[commandchars=\\\{\}]
{\color{incolor}In [{\color{incolor}15}]:} \PY{n+nb}{print}\PY{p}{(}\PY{n}{freq}\PY{o}{.}\PY{n}{shape}\PY{p}{,} \PY{n}{period1}\PY{o}{.}\PY{n}{shape}\PY{p}{)}
\end{Verbatim}


    \begin{Verbatim}[commandchars=\\\{\}]
(501,) (36,)

    \end{Verbatim}

    Nous constatons que le spectre en fréquence de Fourier est plus précis
et défini en fréquence que le spectre des wavelets. Nous nous attendions
à un tel résultats : en effet, afin de conserver la résolution
temporelle, la transformée en ondelettes sacrifie sa résolution
fréquentielle.\\
Ainsi, pour la transformée de Fourier, \(\Delta f\) a pour valeur :

    \begin{Verbatim}[commandchars=\\\{\}]
{\color{incolor}In [{\color{incolor}16}]:} \PY{n+nb}{print}\PY{p}{(}\PY{l+s+s1}{\PYZsq{}}\PY{l+s+s1}{Delta\PYZus{}f = }\PY{l+s+s1}{\PYZsq{}}\PY{p}{,} \PY{n}{freq}\PY{p}{[}\PY{l+m+mi}{42}\PY{p}{]} \PY{o}{\PYZhy{}} \PY{n}{freq}\PY{p}{[}\PY{l+m+mi}{41}\PY{p}{]}\PY{p}{,} \PY{l+s+s1}{\PYZsq{}}\PY{l+s+s1}{Hz}\PY{l+s+s1}{\PYZsq{}}\PY{p}{)}
\end{Verbatim}


    \begin{Verbatim}[commandchars=\\\{\}]
Delta\_f =  0.10000000000000053 Hz

    \end{Verbatim}

    Alors que pour la transformée en ondelettes, cette valeur est variable,
comme nous pouvons le voir dans l'array des coordonnées fréquentielles
(obtenue avec \(1/f \simeq T\)) :

    \begin{Verbatim}[commandchars=\\\{\}]
{\color{incolor}In [{\color{incolor}17}]:} \PY{n+nb}{print}\PY{p}{(}\PY{l+m+mi}{1}\PY{o}{/}\PY{n}{period1}\PY{p}{)}
\end{Verbatim}


    \begin{Verbatim}[commandchars=\\\{\}]
[48.40066546 40.69994608 34.22443876 28.77920787 24.20033273 20.34997304
 17.11221938 14.38960393 12.10016636 10.17498652  8.55610969  7.19480197
  6.05008318  5.08749326  4.27805485  3.59740098  3.02504159  2.54374663
  2.13902742  1.79870049  1.5125208   1.27187332  1.06951371  0.89935025
  0.7562604   0.63593666  0.53475686  0.44967512  0.3781302   0.31796833
  0.26737843  0.22483756  0.1890651   0.15898416  0.13368921  0.11241878]

    \end{Verbatim}

    Comme la plage de fréquences parcourue par les deux transformées reste
proche :

    \begin{Verbatim}[commandchars=\\\{\}]
{\color{incolor}In [{\color{incolor}18}]:} \PY{n+nb}{print}\PY{p}{(}\PY{l+s+s1}{\PYZsq{}}\PY{l+s+s1}{Fourier :   }\PY{l+s+s1}{\PYZsq{}}\PY{p}{,} \PY{n}{freq}\PY{p}{[}\PY{o}{\PYZhy{}}\PY{l+m+mi}{1}\PY{p}{]} \PY{o}{\PYZhy{}} \PY{n}{freq}\PY{p}{[}\PY{l+m+mi}{0}\PY{p}{]}\PY{p}{,} \PY{l+s+s1}{\PYZsq{}}\PY{l+s+s1}{Hz}\PY{l+s+s1}{\PYZsq{}}\PY{p}{)}
         \PY{n+nb}{print}\PY{p}{(}\PY{l+s+s1}{\PYZsq{}}\PY{l+s+s1}{Ondelettes :}\PY{l+s+s1}{\PYZsq{}}\PY{p}{,} \PY{l+m+mi}{1}\PY{o}{/}\PY{n}{period1}\PY{p}{[}\PY{l+m+mi}{0}\PY{p}{]} \PY{o}{\PYZhy{}} \PY{l+m+mi}{1}\PY{o}{/}\PY{n}{period1}\PY{p}{[}\PY{o}{\PYZhy{}}\PY{l+m+mi}{1}\PY{p}{]}\PY{p}{,} \PY{l+s+s1}{\PYZsq{}}\PY{l+s+s1}{Hz}\PY{l+s+s1}{\PYZsq{}}\PY{p}{)}
\end{Verbatim}


    \begin{Verbatim}[commandchars=\\\{\}]
Fourier :    50.0 Hz
Ondelettes : 48.28824667898603 Hz

    \end{Verbatim}

    Soit en moyenne \(\Delta f = 48,29/36 = 1,34\) Hz, donc un ordre de
grandeur supérieur au pas de la transformée de Fourier.\\
C'est la conséquence du principe d'incertitude de Heisenberg, qui relie
la résolution en fréquence \(\Delta f\) (ou en échelle temporelle
\(\Delta \tau = 1/\Delta f\)) et la résolution temporelle \(\Delta T\) :
\[ \Delta f \Delta T \sim const.\]

    Nous allons maintenant mettre en application ces transformations sur des
signaux réels.

    \hypertarget{application-uxe0-des-signaux-ruxe9els}{%
\section{Application à des signaux
réels}\label{application-uxe0-des-signaux-ruxe9els}}

    Nous allons maintenant étudier des données spatiales provenant de la
mission CLUSTER lancée en 2000. Elle est constituée de quatre satellites
qui ont pour missions d'étudier l'interaction entre le vent solaire et
la magnétoshpère terrestre en mesurant notamment le champ magnétique en
trois dimensions grâce à la répartition spatiale des quatres
satellites.\\
Aujourd'hui, nous nous intéressons au relevé des mesures de champs
magnétique du 31 mars 2001 qui présente de multiples traversées de la
zone de choc entre le vent solaire et la magnétosphère terrestre.

En utilisant un module de \texttt{scipy} pour lire le fichier en
\texttt{.sav}, nous pouvons tracer ces données et afin d'avoir une
meilleure idée de ce qu'il se passe, calculons le module du vecteur
champ magnétique :

    \begin{Verbatim}[commandchars=\\\{\}]
{\color{incolor}In [{\color{incolor}19}]:} \PY{k+kn}{from} \PY{n+nn}{scipy}\PY{n+nn}{.}\PY{n+nn}{io} \PY{k}{import} \PY{n}{readsav}
         \PY{n}{data} \PY{o}{=} \PY{n}{readsav}\PY{p}{(}\PY{l+s+s1}{\PYZsq{}}\PY{l+s+s1}{data\PYZus{}choc.sav}\PY{l+s+s1}{\PYZsq{}}\PY{p}{,} \PY{n}{verbose}\PY{o}{=}\PY{k+kc}{False}\PY{p}{)}   \PY{c+c1}{\PYZsh{} Reading data from .sav file}
\end{Verbatim}


    \begin{Verbatim}[commandchars=\\\{\}]
{\color{incolor}In [{\color{incolor}20}]:} \PY{n}{t} \PY{o}{=} \PY{n}{data}\PY{p}{[}\PY{l+s+s1}{\PYZsq{}}\PY{l+s+s1}{t\PYZus{}sc}\PY{l+s+s1}{\PYZsq{}}\PY{p}{]}
         \PY{n}{Bx} \PY{o}{=} \PY{n}{data}\PY{p}{[}\PY{l+s+s1}{\PYZsq{}}\PY{l+s+s1}{bx}\PY{l+s+s1}{\PYZsq{}}\PY{p}{]}
         \PY{n}{By} \PY{o}{=} \PY{n}{data}\PY{p}{[}\PY{l+s+s1}{\PYZsq{}}\PY{l+s+s1}{by}\PY{l+s+s1}{\PYZsq{}}\PY{p}{]}
         \PY{n}{Bz} \PY{o}{=} \PY{n}{data}\PY{p}{[}\PY{l+s+s1}{\PYZsq{}}\PY{l+s+s1}{bz}\PY{l+s+s1}{\PYZsq{}}\PY{p}{]}
         \PY{n}{dt} \PY{o}{=} \PY{n}{t}\PY{p}{[}\PY{l+m+mi}{42}\PY{p}{]} \PY{o}{\PYZhy{}} \PY{n}{t}\PY{p}{[}\PY{l+m+mi}{41}\PY{p}{]}    \PY{c+c1}{\PYZsh{} Timestep}
         \PY{n}{K} \PY{o}{=} \PY{l+m+mi}{3600}              \PY{c+c1}{\PYZsh{} Constant for seconds to decimal hours conversion}
         \PY{n}{mod} \PY{o}{=} \PY{n}{np}\PY{o}{.}\PY{n}{sqrt}\PY{p}{(}\PY{n}{Bx}\PY{o}{*}\PY{o}{*}\PY{l+m+mi}{2} \PY{o}{+} \PY{n}{By}\PY{o}{*}\PY{o}{*}\PY{l+m+mi}{2} \PY{o}{+} \PY{n}{Bz}\PY{o}{*}\PY{o}{*}\PY{l+m+mi}{2}\PY{p}{)}     \PY{c+c1}{\PYZsh{} Computing the modulus}
\end{Verbatim}


    \begin{Verbatim}[commandchars=\\\{\}]
{\color{incolor}In [{\color{incolor}21}]:} \PY{n}{plt}\PY{o}{.}\PY{n}{subplot}\PY{p}{(}\PY{l+m+mi}{211}\PY{p}{)}
         \PY{n}{plt}\PY{o}{.}\PY{n}{plot}\PY{p}{(}\PY{n}{t}\PY{o}{/}\PY{n}{K}\PY{p}{,} \PY{n}{Bx}\PY{p}{,} \PY{n}{label}\PY{o}{=}\PY{l+s+sa}{r}\PY{l+s+s1}{\PYZsq{}}\PY{l+s+s1}{\PYZdl{}B\PYZus{}x\PYZdl{}}\PY{l+s+s1}{\PYZsq{}}\PY{p}{,} \PY{n}{linewidth}\PY{o}{=}\PY{l+m+mf}{1.2}\PY{p}{)}
         \PY{n}{plt}\PY{o}{.}\PY{n}{plot}\PY{p}{(}\PY{n}{t}\PY{o}{/}\PY{n}{K}\PY{p}{,} \PY{n}{By}\PY{p}{,} \PY{n}{label}\PY{o}{=}\PY{l+s+sa}{r}\PY{l+s+s1}{\PYZsq{}}\PY{l+s+s1}{\PYZdl{}B\PYZus{}y\PYZdl{}}\PY{l+s+s1}{\PYZsq{}}\PY{p}{,} \PY{n}{linewidth}\PY{o}{=}\PY{l+m+mf}{1.2}\PY{p}{)}
         \PY{n}{plt}\PY{o}{.}\PY{n}{plot}\PY{p}{(}\PY{n}{t}\PY{o}{/}\PY{n}{K}\PY{p}{,} \PY{n}{Bz}\PY{p}{,} \PY{n}{label}\PY{o}{=}\PY{l+s+sa}{r}\PY{l+s+s1}{\PYZsq{}}\PY{l+s+s1}{\PYZdl{}B\PYZus{}z\PYZdl{}}\PY{l+s+s1}{\PYZsq{}}\PY{p}{,} \PY{n}{linewidth}\PY{o}{=}\PY{l+m+mf}{1.2}\PY{p}{)}
         \PY{n}{plt}\PY{o}{.}\PY{n}{xlim}\PY{p}{(}\PY{p}{(}\PY{n}{np}\PY{o}{.}\PY{n}{min}\PY{p}{(}\PY{n}{t}\PY{o}{/}\PY{n}{K}\PY{p}{)}\PY{p}{,} \PY{n}{np}\PY{o}{.}\PY{n}{max}\PY{p}{(}\PY{n}{t}\PY{o}{/}\PY{n}{K}\PY{p}{)}\PY{p}{)}\PY{p}{)}
         \PY{n}{plt}\PY{o}{.}\PY{n}{xlabel}\PY{p}{(}\PY{l+s+s1}{\PYZsq{}}\PY{l+s+s1}{Temps (h.dec)}\PY{l+s+s1}{\PYZsq{}}\PY{p}{)}
         \PY{n}{plt}\PY{o}{.}\PY{n}{ylabel}\PY{p}{(}\PY{l+s+s1}{\PYZsq{}}\PY{l+s+s1}{Magnetic field (nT)}\PY{l+s+s1}{\PYZsq{}}\PY{p}{)}
         \PY{n}{plt}\PY{o}{.}\PY{n}{legend}\PY{p}{(}\PY{p}{)}
         
         \PY{n}{plt}\PY{o}{.}\PY{n}{subplot}\PY{p}{(}\PY{l+m+mi}{212}\PY{p}{)}
         \PY{n}{plt}\PY{o}{.}\PY{n}{plot}\PY{p}{(}\PY{n}{t}\PY{o}{/}\PY{n}{K}\PY{p}{,} \PY{n}{mod}\PY{p}{,} \PY{n}{label}\PY{o}{=}\PY{l+s+sa}{r}\PY{l+s+s1}{\PYZsq{}}\PY{l+s+s1}{\PYZdl{}}\PY{l+s+s1}{\PYZbs{}}\PY{l+s+s1}{left| }\PY{l+s+s1}{\PYZbs{}}\PY{l+s+s1}{mathbf}\PY{l+s+si}{\PYZob{}B\PYZcb{}}\PY{l+s+s1}{\PYZbs{}}\PY{l+s+s1}{right|\PYZdl{}}\PY{l+s+s1}{\PYZsq{}}\PY{p}{,} \PY{n}{linewidth}\PY{o}{=}\PY{l+m+mf}{1.2}\PY{p}{)}
         \PY{n}{plt}\PY{o}{.}\PY{n}{xlim}\PY{p}{(}\PY{p}{(}\PY{n}{np}\PY{o}{.}\PY{n}{min}\PY{p}{(}\PY{n}{t}\PY{o}{/}\PY{n}{K}\PY{p}{)}\PY{p}{,} \PY{n}{np}\PY{o}{.}\PY{n}{max}\PY{p}{(}\PY{n}{t}\PY{o}{/}\PY{n}{K}\PY{p}{)}\PY{p}{)}\PY{p}{)}
         \PY{n}{plt}\PY{o}{.}\PY{n}{xlabel}\PY{p}{(}\PY{l+s+s1}{\PYZsq{}}\PY{l+s+s1}{Time (h.dec)}\PY{l+s+s1}{\PYZsq{}}\PY{p}{)}
         \PY{n}{plt}\PY{o}{.}\PY{n}{ylabel}\PY{p}{(}\PY{l+s+s1}{\PYZsq{}}\PY{l+s+s1}{Magnetic field (nT)}\PY{l+s+s1}{\PYZsq{}}\PY{p}{)}
         \PY{n}{plt}\PY{o}{.}\PY{n}{legend}\PY{p}{(}\PY{p}{)}
         \PY{n}{plt}\PY{o}{.}\PY{n}{tight\PYZus{}layout}\PY{p}{(}\PY{p}{)}
\end{Verbatim}


    \begin{center}
    \adjustimage{max size={0.9\linewidth}{0.9\paperheight}}{Rapport_files/Rapport_41_0.png}
    \end{center}
    { \hspace*{\fill} \\}
    
    Nous remarquons de multiples franchissements de la zone de choc marqués
par les discontinuités : en effet, cette zone est située à l'équilibre
entre la magnétosphère et le vent solaire dont la vitesse varie, donc sa
position varie également. Elle correspond à une onde de choc non
collisionnelle entre le vent solaire qui est supersonique
(\(v_{sw} \simeq 500\) km/s alors que la vitesse du son dans le milieu
est de \(v_{m,son} \simeq 70\) km/s) et le milieu de la magnétogaine
terrestre où il devient subsonique. Ainsi, comme sa vitesse diminue,
l'énergie cinétique est convertie en énergie magnétique, d'où
l'augmentation du module du champ magnétique dans la magnétospère de
\(\left| \mathbf{B}\right| \simeq 30\) nT à
\(\left| \mathbf{B}\right| \simeq 90\) nT.

Dans cette étude, nous allons nous concentrer sur un unique
franchissement de la zone de choc, par exemple entre 17,8 h et 18,2 h.
Réalisons ensuite la transformée de Fourier de cet intervalle (en
utilisant une \texttt{fft} pour des raisons de rapidité et d'usage de
mémoire avec \texttt{fdft}).

    \begin{Verbatim}[commandchars=\\\{\}]
{\color{incolor}In [{\color{incolor}22}]:} \PY{n}{mask} \PY{o}{=} \PY{p}{(}\PY{p}{(}\PY{n}{t} \PY{o}{\PYZlt{}} \PY{l+m+mf}{18.2}\PY{o}{*}\PY{n}{K}\PY{p}{)} \PY{o}{\PYZam{}} \PY{p}{(}\PY{n}{t} \PY{o}{\PYZgt{}} \PY{l+m+mf}{17.8}\PY{o}{*}\PY{n}{K}\PY{p}{)}\PY{p}{)}        \PY{c+c1}{\PYZsh{} Mask to select the correct data}
         \PY{n}{T} \PY{o}{=} \PY{n}{t}\PY{p}{[}\PY{n}{mask}\PY{p}{]}\PY{p}{[}\PY{o}{\PYZhy{}}\PY{l+m+mi}{1}\PY{p}{]} \PY{o}{\PYZhy{}} \PY{n}{t}\PY{p}{[}\PY{n}{mask}\PY{p}{]}\PY{p}{[}\PY{l+m+mi}{0}\PY{p}{]}                \PY{c+c1}{\PYZsh{} Duration of the selected range}
         \PY{n}{PSDa0} \PY{o}{=} \PY{l+m+mi}{2} \PY{o}{*} \PY{n}{T} \PY{o}{*} \PY{n}{np}\PY{o}{.}\PY{n}{abs}\PY{p}{(}\PY{n}{np}\PY{o}{.}\PY{n}{fft}\PY{o}{.}\PY{n}{rfft}\PY{p}{(}\PY{n}{mod}\PY{p}{[}\PY{n}{mask}\PY{p}{]}\PY{p}{)}\PY{p}{)}\PY{o}{*}\PY{o}{*}\PY{l+m+mi}{2}
         \PY{n}{freqa0} \PY{o}{=} \PY{n}{np}\PY{o}{.}\PY{n}{fft}\PY{o}{.}\PY{n}{rfftfreq}\PY{p}{(}\PY{n+nb}{len}\PY{p}{(}\PY{n}{mod}\PY{p}{[}\PY{n}{mask}\PY{p}{]}\PY{p}{)}\PY{p}{,} \PY{n}{dt}\PY{p}{)}
\end{Verbatim}


    \begin{Verbatim}[commandchars=\\\{\}]
{\color{incolor}In [{\color{incolor}23}]:} \PY{n}{plt}\PY{o}{.}\PY{n}{subplot}\PY{p}{(}\PY{l+m+mi}{211}\PY{p}{)}
         \PY{n}{plt}\PY{o}{.}\PY{n}{plot}\PY{p}{(}\PY{n}{t}\PY{p}{[}\PY{n}{mask}\PY{p}{]}\PY{o}{/}\PY{n}{K}\PY{p}{,} \PY{n}{mod}\PY{p}{[}\PY{n}{mask}\PY{p}{]}\PY{p}{,} \PY{n}{label}\PY{o}{=}\PY{l+s+sa}{r}\PY{l+s+s1}{\PYZsq{}}\PY{l+s+s1}{\PYZdl{}}\PY{l+s+s1}{\PYZbs{}}\PY{l+s+s1}{left| }\PY{l+s+s1}{\PYZbs{}}\PY{l+s+s1}{mathbf}\PY{l+s+si}{\PYZob{}B\PYZcb{}}\PY{l+s+s1}{\PYZbs{}}\PY{l+s+s1}{right|\PYZdl{}}\PY{l+s+s1}{\PYZsq{}}\PY{p}{,} \PY{n}{linewidth}\PY{o}{=}\PY{l+m+mf}{1.2}\PY{p}{)}
         \PY{n}{plt}\PY{o}{.}\PY{n}{xlim}\PY{p}{(}\PY{p}{(}\PY{n}{np}\PY{o}{.}\PY{n}{min}\PY{p}{(}\PY{n}{t}\PY{p}{[}\PY{n}{mask}\PY{p}{]}\PY{o}{/}\PY{n}{K}\PY{p}{)}\PY{p}{,} \PY{n}{np}\PY{o}{.}\PY{n}{max}\PY{p}{(}\PY{n}{t}\PY{p}{[}\PY{n}{mask}\PY{p}{]}\PY{o}{/}\PY{n}{K}\PY{p}{)}\PY{p}{)}\PY{p}{)}
         \PY{n}{plt}\PY{o}{.}\PY{n}{xlabel}\PY{p}{(}\PY{l+s+s1}{\PYZsq{}}\PY{l+s+s1}{Time (h.dec)}\PY{l+s+s1}{\PYZsq{}}\PY{p}{)}
         \PY{n}{plt}\PY{o}{.}\PY{n}{ylabel}\PY{p}{(}\PY{l+s+s1}{\PYZsq{}}\PY{l+s+s1}{Magnetic field (nT)}\PY{l+s+s1}{\PYZsq{}}\PY{p}{)}
         \PY{n}{plt}\PY{o}{.}\PY{n}{legend}\PY{p}{(}\PY{p}{)}
         
         \PY{n}{plt}\PY{o}{.}\PY{n}{subplot}\PY{p}{(}\PY{l+m+mi}{212}\PY{p}{)}
         \PY{n}{plt}\PY{o}{.}\PY{n}{loglog}\PY{p}{(}\PY{n}{freqa0}\PY{p}{,} \PY{n}{PSDa0}\PY{p}{,} \PY{n}{linewidth}\PY{o}{=}\PY{l+m+mf}{1.2}\PY{p}{)}
         \PY{n}{plt}\PY{o}{.}\PY{n}{xlim}\PY{p}{(}\PY{p}{(}\PY{n}{np}\PY{o}{.}\PY{n}{min}\PY{p}{(}\PY{n}{freqa0}\PY{p}{)}\PY{p}{,} \PY{n}{np}\PY{o}{.}\PY{n}{max}\PY{p}{(}\PY{n}{freqa0}\PY{p}{)}\PY{p}{)}\PY{p}{)}
         \PY{n}{plt}\PY{o}{.}\PY{n}{xlabel}\PY{p}{(}\PY{l+s+s1}{\PYZsq{}}\PY{l+s+s1}{Frequency (Hz)}\PY{l+s+s1}{\PYZsq{}}\PY{p}{)}
         \PY{n}{plt}\PY{o}{.}\PY{n}{ylabel}\PY{p}{(}\PY{l+s+s1}{\PYZsq{}}\PY{l+s+s1}{PSD (V\PYZca{}2 / Hz)}\PY{l+s+s1}{\PYZsq{}}\PY{p}{)}
         \PY{n}{plt}\PY{o}{.}\PY{n}{tight\PYZus{}layout}\PY{p}{(}\PY{p}{)}
\end{Verbatim}


    \begin{center}
    \adjustimage{max size={0.9\linewidth}{0.9\paperheight}}{Rapport_files/Rapport_44_0.png}
    \end{center}
    { \hspace*{\fill} \\}
    
    Le pas de temps limite la gamme de fréquence couverte à
\(\Delta f = 12,8\) Hz par théorème de Shannon-Nyquist, et le bruit de
la transformée nous empêche de voir les structures fréquentielles
clairement. Appliquons la transformée en ondelettes sur cette plage de
données.

    \begin{Verbatim}[commandchars=\\\{\}]
{\color{incolor}In [{\color{incolor}24}]:} \PY{n}{wavea0}\PY{p}{,} \PY{n}{perioda0}\PY{p}{,} \PY{n}{scalea0}\PY{p}{,} \PY{n}{coia0} \PY{o}{=} \PY{n}{wav}\PY{o}{.}\PY{n}{wavelet}\PY{p}{(}\PY{n}{mod}\PY{p}{[}\PY{n}{mask}\PY{p}{]}\PY{p}{,} \PY{n}{dt}\PY{p}{)}
\end{Verbatim}


    \begin{Verbatim}[commandchars=\\\{\}]
{\color{incolor}In [{\color{incolor}25}]:} \PY{n}{contour\PYZus{}plot} \PY{o}{=} \PY{n}{plt}\PY{o}{.}\PY{n}{contourf}\PY{p}{(}\PY{n}{t}\PY{p}{[}\PY{n}{mask}\PY{p}{]}\PY{o}{/}\PY{n}{K}\PY{p}{,} \PY{n}{perioda0}\PY{p}{,} \PY{n}{np}\PY{o}{.}\PY{n}{log10}\PY{p}{(}\PY{n}{np}\PY{o}{.}\PY{n}{abs}\PY{p}{(}\PY{n}{wavea0}\PY{p}{)}\PY{o}{*}\PY{o}{*}\PY{l+m+mi}{2}\PY{p}{)}\PY{p}{,} \PY{l+m+mi}{42}\PY{p}{,} \PY{n}{extend}\PY{o}{=}\PY{l+s+s1}{\PYZsq{}}\PY{l+s+s1}{both}\PY{l+s+s1}{\PYZsq{}}\PY{p}{)}
         \PY{n}{plt}\PY{o}{.}\PY{n}{plot}\PY{p}{(}\PY{n}{t}\PY{p}{[}\PY{n}{mask}\PY{p}{]}\PY{o}{/}\PY{n}{K}\PY{p}{,} \PY{n}{coia0}\PY{p}{,} \PY{n}{color}\PY{o}{=}\PY{l+s+s1}{\PYZsq{}}\PY{l+s+s1}{red}\PY{l+s+s1}{\PYZsq{}}\PY{p}{)}
         \PY{n}{plt}\PY{o}{.}\PY{n}{ylim}\PY{p}{(}\PY{p}{(}\PY{n}{np}\PY{o}{.}\PY{n}{min}\PY{p}{(}\PY{n}{perioda0}\PY{p}{)}\PY{p}{,} \PY{n}{np}\PY{o}{.}\PY{n}{max}\PY{p}{(}\PY{n}{perioda0}\PY{p}{)}\PY{p}{)}\PY{p}{)}
         \PY{n}{plt}\PY{o}{.}\PY{n}{yscale}\PY{p}{(}\PY{l+s+s1}{\PYZsq{}}\PY{l+s+s1}{log}\PY{l+s+s1}{\PYZsq{}}\PY{p}{)}
         \PY{n}{plt}\PY{o}{.}\PY{n}{xlabel}\PY{p}{(}\PY{l+s+s1}{\PYZsq{}}\PY{l+s+s1}{Time (h.dec)}\PY{l+s+s1}{\PYZsq{}}\PY{p}{)}
         \PY{n}{plt}\PY{o}{.}\PY{n}{ylabel}\PY{p}{(}\PY{l+s+sa}{r}\PY{l+s+s1}{\PYZsq{}}\PY{l+s+s1}{Period (s)}\PY{l+s+s1}{\PYZsq{}}\PY{p}{)}
         \PY{n}{plt}\PY{o}{.}\PY{n}{title}\PY{p}{(}\PY{l+s+sa}{r}\PY{l+s+s1}{\PYZsq{}}\PY{l+s+s1}{Scaleogram of the crossing of a shock region}\PY{l+s+s1}{\PYZsq{}}\PY{p}{)}
         \PY{n}{cbar} \PY{o}{=} \PY{n}{plt}\PY{o}{.}\PY{n}{colorbar}\PY{p}{(}\PY{n}{contour\PYZus{}plot}\PY{p}{)}
         \PY{n}{cbar}\PY{o}{.}\PY{n}{set\PYZus{}label}\PY{p}{(}\PY{l+s+sa}{r}\PY{l+s+s1}{\PYZsq{}}\PY{l+s+s1}{\PYZdl{}}\PY{l+s+s1}{\PYZbs{}}\PY{l+s+s1}{log }\PY{l+s+s1}{\PYZbs{}}\PY{l+s+s1}{left( }\PY{l+s+s1}{\PYZbs{}}\PY{l+s+s1}{left| }\PY{l+s+s1}{\PYZbs{}}\PY{l+s+s1}{mathcal}\PY{l+s+si}{\PYZob{}W\PYZcb{}}\PY{l+s+s1}{(}\PY{l+s+s1}{\PYZbs{}}\PY{l+s+s1}{tau,t)}\PY{l+s+s1}{\PYZbs{}}\PY{l+s+s1}{right| \PYZca{}}\PY{l+s+si}{\PYZob{}2\PYZcb{}}\PY{l+s+s1}{\PYZbs{}}\PY{l+s+s1}{right)\PYZdl{}}\PY{l+s+s1}{\PYZsq{}}\PY{p}{,} \PY{n}{rotation}\PY{o}{=}\PY{l+m+mi}{270}\PY{p}{,} \PY{n}{labelpad}\PY{o}{=}\PY{l+m+mi}{30}\PY{p}{)}
         \PY{n}{plt}\PY{o}{.}\PY{n}{show}\PY{p}{(}\PY{p}{)}
\end{Verbatim}


    \begin{center}
    \adjustimage{max size={0.9\linewidth}{0.9\paperheight}}{Rapport_files/Rapport_47_0.png}
    \end{center}
    { \hspace*{\fill} \\}
    
    Dans ce scalogramme, nous voyons apparaître des structures verticales,
significative de phénomènes localisés dans le temps mettant en jeu
toutes les échelles de période (contrairement aux ondes de la partie
précédente, ayant une période fixée). Nous observons aussi l'existence
de valeurs non physiques en-dehors du \emph{cone of influence} (COI)
tracé en rouge sur la figure ci-dessus.\\
Comme nous nous y attendions, la gamme d'échelles temporelles résolues
est plus petite que pour la transformée de Fourier (par principe
d'incertitude d'Heisenberg) :
\[\Delta f_{wavelet} = f_{max} - f_{min} = \frac{1}{T_{min}} - \frac{1}{T_{max}} = 12,39 ~\text{Hz}\]
Pour obtenir ce résultat, nous utilisons le fait que \(1/f \simeq T\)
dans notre cas.

    \begin{Verbatim}[commandchars=\\\{\}]
{\color{incolor}In [{\color{incolor}26}]:} \PY{n+nb}{print}\PY{p}{(}\PY{l+m+mi}{1}\PY{o}{/}\PY{n}{perioda0}\PY{p}{[}\PY{l+m+mi}{0}\PY{p}{]} \PY{o}{\PYZhy{}} \PY{l+m+mi}{1}\PY{o}{/}\PY{n}{perioda0}\PY{p}{[}\PY{o}{\PYZhy{}}\PY{l+m+mi}{1}\PY{p}{]}\PY{p}{)}
\end{Verbatim}


    \begin{Verbatim}[commandchars=\\\{\}]
12.3898140972904

    \end{Verbatim}

    Nous observons bien une densité d'énergie plus grande dans la zone de
gauche qui correspond au champ magnétique dans la magnétogaine que dans
la zone du vent solaire. Entre ces deux zones, l'onde de choc est
caractérisée par sa localisation temporelle et la mise en jeu de toutes
les échelles temporelles du plasma. Tout l'intérêt de la transformée en
ondelettes est ici exploité : nous pouvons localiser dans le temps les
différents modes temporels présents.

Comparons enfin le spectre de Fourier avec le spectre des wavelets :

    \begin{Verbatim}[commandchars=\\\{\}]
{\color{incolor}In [{\color{incolor}27}]:} \PY{n}{spWa0} \PY{o}{=} \PY{n}{np}\PY{o}{.}\PY{n}{sum}\PY{p}{(}\PY{n}{np}\PY{o}{.}\PY{n}{abs}\PY{p}{(}\PY{n}{wavea0}\PY{p}{)}\PY{o}{*}\PY{o}{*}\PY{l+m+mi}{2}\PY{p}{,} \PY{n}{axis}\PY{o}{=}\PY{l+m+mi}{1}\PY{p}{)}\PY{o}{/}\PY{n}{wavea0}\PY{o}{.}\PY{n}{shape}\PY{p}{[}\PY{l+m+mi}{1}\PY{p}{]}
\end{Verbatim}


    \begin{Verbatim}[commandchars=\\\{\}]
{\color{incolor}In [{\color{incolor}28}]:} \PY{n}{plt}\PY{o}{.}\PY{n}{figure}\PY{p}{(}\PY{n}{figsize}\PY{o}{=}\PY{p}{(}\PY{l+m+mi}{17}\PY{p}{,}\PY{l+m+mi}{5}\PY{p}{)}\PY{p}{)}
         \PY{n}{plt}\PY{o}{.}\PY{n}{loglog}\PY{p}{(}\PY{n}{freqa0}\PY{p}{,} \PY{n}{PSDa0}\PY{p}{,} \PY{n}{linewidth}\PY{o}{=}\PY{l+m+mf}{1.2}\PY{p}{,} \PY{n}{label}\PY{o}{=}\PY{l+s+s1}{\PYZsq{}}\PY{l+s+s1}{Fourier}\PY{l+s+s1}{\PYZsq{}}\PY{p}{)}
         \PY{n}{plt}\PY{o}{.}\PY{n}{loglog}\PY{p}{(}\PY{l+m+mi}{1}\PY{o}{/}\PY{n}{perioda0}\PY{p}{,} \PY{n}{spWa0}\PY{p}{,} \PY{n}{linewidth}\PY{o}{=}\PY{l+m+mf}{1.2}\PY{p}{,} \PY{n}{label}\PY{o}{=}\PY{l+s+s1}{\PYZsq{}}\PY{l+s+s1}{Wavelets}\PY{l+s+s1}{\PYZsq{}}\PY{p}{)}
         \PY{n}{plt}\PY{o}{.}\PY{n}{xlim}\PY{p}{(}\PY{p}{(}\PY{n}{np}\PY{o}{.}\PY{n}{min}\PY{p}{(}\PY{n}{freqa0}\PY{p}{)}\PY{p}{,} \PY{n}{np}\PY{o}{.}\PY{n}{max}\PY{p}{(}\PY{n}{freqa0}\PY{p}{)}\PY{p}{)}\PY{p}{)}
         \PY{n}{plt}\PY{o}{.}\PY{n}{xlabel}\PY{p}{(}\PY{l+s+s1}{\PYZsq{}}\PY{l+s+s1}{Frequency (Hz)}\PY{l+s+s1}{\PYZsq{}}\PY{p}{)}
         \PY{n}{plt}\PY{o}{.}\PY{n}{ylabel}\PY{p}{(}\PY{l+s+s1}{\PYZsq{}}\PY{l+s+s1}{PSD (V\PYZca{}2 / Hz)}\PY{l+s+s1}{\PYZsq{}}\PY{p}{)}
         \PY{n}{plt}\PY{o}{.}\PY{n}{legend}\PY{p}{(}\PY{p}{)}
         \PY{n}{plt}\PY{o}{.}\PY{n}{show}\PY{p}{(}\PY{p}{)}
\end{Verbatim}


    \begin{center}
    \adjustimage{max size={0.9\linewidth}{0.9\paperheight}}{Rapport_files/Rapport_52_0.png}
    \end{center}
    { \hspace*{\fill} \\}
    
    Nous constatons de la même façon que précédemment que la résolution en
fréquence de la transformée en ondelettes est bien moins grande que la
transformée de Fourier, mais elle nous permet de conserver l'information
sur la localisation temporelle des modes du signal. Les deux sont donc
très efficaces dans leurs utilisations respectives, et je dirais même
plus, complémentaires. Il est d'ailleurs tout aussi simple d'interpréter
les résultats des deux transformées (avec un léger avantage pour Fourier
tout de même).\\
Par contre, la transformée en ondelettes a une précision en basse
fréquences plus faible que la transformée de Fourier.

    \hypertarget{conclusion}{%
\section{Conclusion}\label{conclusion}}

Au cours de ce TP, nous avons introduit la transformée de Fourier via
une DFT manuelle, ainsi que la transformée en ondelettes de Morlet. Nous
avons pu exhiber la différence entre les deux, et leur application à des
données réelles nous a permis de montrer leurs domaines d'applications,
leurs forces et leurs faiblesses. Finalement, nous ne pouvons que dire
que ces deux approches sont complémentaires lors de l'étude d'un signal
temporel, car elles permettent de mettre en évidence des phénomènes
différents.

Je tiens à remercier Olga Alexandrova et Baptiste Cecconi du
\href{http://lesia.obspm.fr/}{LESIA} pour leur encadrement durant cette
journée et leur passion communicative. Merci beaucoup !


    % Add a bibliography block to the postdoc
    
    
    
    \end{document}
